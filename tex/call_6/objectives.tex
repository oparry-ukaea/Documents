
One of the key-aims of \nep \   work under FM-WP2~\cite{sciplan} is to explore options for 
scalable, Exascale targeted closed form models, based upon a fluid treatment of 
hot, thermonuclear plasma at the edge of a tokamak device (complementary to a 
gyroaveraged treatment). The fluid model must capture the physics of the plasma in 
regions including the separatrix but away from the metal surfaces of the device. 
Such a model is necessary to form a core building block of a full, ``actionable'' 
simulation of the tokamak exhaust region including all relevant Multi-physics phenomena 
(around bulk plasma species, impurities and neutrals). To be suitable for analysing 
ITER data and for designing the commercial powerplants of the future, the eventual 
code must be performant and scalable to Exascale class hardware, must be easy to 
deploy upon different architectures, must be actionable (including treatment of uncertainty)
item and must couple efficiently to the infrastructure developed under 
work package FM-WP3 (Neutral Gas and Impurity Model). In the long term, we also 
anticipate that this model will be coupled to a 5-D gyroaveraged model for the bulk 
plasma that comes into direct contact with plasma facing components of the device. 
The model and its instantiation in the eventual code must therefore take all of 
these requirements into account.

The desired properties of the ideal referent fluid model include the following:

\begin{itemize}
\item[\textbf{PR1}] Ability to capture the physics (including both electrons and ions) 
of the bulk plasma using fluid theory. It should be borne in mind that the electrons 
and ions have very disparate length and time scales; the model must be able to 
exploit Exascale computing technology to resolve this computational challenge at 
scale. Given that future Exascale architectures are not well defined, it is strongly 
advised that ``performance portability'' be built into the code base.

\item[\textbf{PR2}] The model must be agile, modular and easy to adapt. This is to ensure 
that it can be coupled to a gyroaveraged model that might be used for other parts 
of the plasma that are eg.\ in direct contact with metal surfaces and to allow 
new physics to be added in a simple, agile way. Solutions around coupling methodology 
and technology should therefore be considered, as should the use of Domain Specific 
Language (DSL) and code-generator technologies.

\item[\textbf{PR3}] Later in the project, the model will likely be coupled to a particle 
model for neutral and impurity species. As per the gyroaveraged model in PR2, this must also 
be taken into account in the design of the model/algorithm; this will require close 
cooperation/co-design with other team members in order to design the particle data 
structures accordingly.

\item[\textbf{PR4}] If a moments-based approach is to be used, closure eg.\ \cite{Ca10Tran,Vo91Flui} as well 
as additional terms such as those describing collisions must be treated correctly. 
The uncertainty resulting from the choice of closure must also eventually be quantified 
and incorporated into the Uncertainty Quantification (UQ) propagation through the 
entire \nep \   edge code/workflow (to ensure that the infrastructure is actionable). 
A comparative analysis of different closure terms is highly encouraged.

\item[\textbf{PR5}] The fluid model within the eventual \nep \   edge code should also 
deliver adequate modelling of 3-D turbulence at edge plasma relevant scales (ie.\  
ideally to be capable of exploiting extreme scalability to directly expose turbulence 
phenomena in the edge and not just parametrize the impact of turbulence). We would 
therefore like the bidder to outline their vision for achieving this end-goal.
\end{itemize}

In addition to the above overarching requirements, the baseline deliverables for 
this call are further specified in the following sections. For each, the bidder 
is requested to provide a clear description of what they will deliver and a plan 
for how they intend to deliver it, including a timeline (taking into account the 
funding available for each financial year specified in section 2.4.2).
%Creativity and ambition for delivering beyond core scope is encouraged.

The task package itself is divided into tasks, each with a set of objectives as 
outlined below. The baseline anticipated outputs and results for each task are 
summarised in section 2.1. 

\begin{itemize}
\item[1.3.1]\textbf{Task 1: Delivery of System 2-2, 2-D elliptic solver in complex geometry}

The ideal numerical algorithms for building the Exascale edge plasma codes of the 
future will encode functionality that delivers efficient and accurate solutions 
of tokamak plasma relevant elliptic problems. The baseline option, which does 
not preclude a consideration of others, has been tentatively identified as ``high 
order'' spectral/hp elements (which it is believed will scale favourably to the Exascale 
hardware due to the compute dense nature of the high order methods~\cite{pappeqs}). An important 
outcome of this stage of the project will be to confirm that high order spectral/hp 
methods are indeed the optimal solution for the eventual code. The bidder should 
therefore indicate how they intend to address this question by working closely 
with the successful bidder to Call~T/NA078/20, potentially by also coding up a lower order 
solver (ideally Hybridised Discontinuous Galerkin (HDG)) within the same \Papp \ 
(or as an additional derivative \Papp ), or by making careful quantified comparisons 
with existing codes/infrastructure. The bidder may also wish to consider a) how 
the impact of state-of-the-art in pre-conditioners will affect the efficacy of 
the different methods and b) whether a ``hybrid'' approach might prevail where 
HDG and Galerkin using spectral/hp elements are operative upon different parts 
of the same grid.

The main objective of this task is to produce an implementation of the model equations 
(primarily using a high order spectral/hp method), geometry and boundary conditions described 
as System~2-2, entitled ``2-D elliptic solver in complex geometry'' of the document
``Equations for \exc/\nep \ \Papp s"~\cite{pappeqs} that will: 

\begin{itemize}
\item[$\bullet$] deliver an ``as accurate as possible solution" that will provide an early indication 
as to whether or not the eventual code based upon this method/solver/library combination will 
cater for problem sizes predicted to deliver an ``actionable solution" on Exascale hardware, 
while giving a ``significant" performance improvement over software using second order representations
without mesh adaptation.
 
\item[$\bullet$] For estimating the likely achievable accuracy, performance and tractable problem sizes/resolution
etc.\ on the eventual Exascale platform, the bidder should indicate how they intend to use the \nep \ \papp(s),
scientific literature, information from trustworthy websites,  analytic theory/solutions, and profiling tools that
perform eg.\  roofline analysis, to identify a viable solution and steer the development towards it.
They should outline their vision (and proposed timeline) for achieving the capability requirements
PR1-5 specified above and in the \exc\ \nep\  Science Plan.


%\item[$\bullet$] Captures as much as possible the scaling characteristics of the full model 
%being targeted in the long term. 
\end{itemize}

\item[1.3.2]\textbf{Task 2: Delivery of System 2-3, 1-D fluid solver with simplified 
physics with UQ and realistic boundary conditions}

As stated in the Fusion Modelling System Science Plan~\cite{sciplan}: 

\begin{itemize}
\item[$\bullet$] Uncertainty quantification (eg.\ via intrusive or ensemble-based methods) 
will be embedded across all areas of the project to ensure~numerical predictions 
are ``actionable''. 

\item[$\bullet$] Identifying, quantifying and mitigating uncertainty and the computational 
overhead of adequately modelling each will be a core theme within the project. 

\item[$\bullet$] The overall aim of the project is to build a hierarchy of models that are 
capable of representing edge plasma behaviour to within a specific level of uncertainty.
\end{itemize}

The main objective of this task is to develop a 1-D fluid solver with simplified 
physics but with UQ and realistic boundary conditions. The aim is to produce an 
implementation of the model equations, geometry and boundary conditions described 
as System~2-3 entitled ``1-D simplified fluid solver with UQ'' of document~\cite{pappeqs} 
that: 

\begin{itemize}
\item[$\bullet$] Delivers an accurate and efficient solution together with a proposal for 
how the method should be extended to higher order \Papp s and the eventual \nep \   
code(s). 

\item[$\bullet$] The bidder should indicate how they intend to assess ``accuracy" and ``efficiency",
together with performance and achievable problem size at Exascale as per Task 1.3.1.

\end{itemize}

\item[1.3.3]\textbf{Task 3: Delivery of System 2-4, 1-D plasma model incorporating velocity
space effects}

As stated in the Fusion Modelling System Science Plan: 

\begin{itemize}
\item[$\bullet$] There may be an inadequate number of collisions for the species to equilibrate, 
leaving significant tails in the particle velocity distributions. 

\item[$\bullet$] Ideas around the discretisation of the velocity (phase) space dependence 
include techniques suitable for dealing with infinite coordinates, such as moment-based 
methods using truncated series and/or mapped finite elements perhaps combined with 
spherical harmonics. 

\item[$\bullet$] The ideal numerical algorithms for forming the Exascale edge plasma codes 
of the future will have preferably at least the following properties: 

\begin{enumerate}
\item Accurate solution of hyperbolic problems. 
\item Accurate representation of velocity~(phase) space. 
\end{enumerate}

\item[$\bullet$] Options, which do not preclude consideration of others, have been tentatively 
identified for initial investigation as follows: spectral/hp elements (high order 
finite elements).
\end{itemize}

The objective of this task is to produce an implementation of the model equations, 
geometry and boundary conditions described as System~2-4 entitled ``1-D plasma 
model incorporating velocity space effects'' of the document~\cite{pappeqs} that: 

\begin{itemize}
\item[$\bullet$] delivers an accurate and efficient solution together with further indications 
around expected performance at higher spatial order and in the eventual code(s). 

\item[$\bullet$] The bidder should indicate how they intend to assess ``accuracy" and ``efficiency",
together with performance and achievable problem size at Exascale as per Task 1.3.1.
\end{itemize}

\item[1.3.4]\textbf{Task 4: Delivery of System 2-5, Spatially 1-D multispecies plasma
model }

As stated in the Fusion Modelling System Science Plan: 

\begin{itemize}
\item[$\bullet$] Plasma electron and ion species can to an extent be treated as separate 
fluids with different temperatures. 

\item[$\bullet$] The plasma in a fusion reactor\ldots will in general contain two main ionic 
species (Deuterium and Tritium), neutral fuel particles and ionised Helium ash 
(or alpha particles), as well as impurity ions originating from the wall. 

\item[$\bullet$] Existing simulations show that attempting to include the\ldots extra physics 
can significantly add to execution cost overhead or a breakdown of scaling, eg.\ 
when stability/accuracy issues are addressed, a large reduction in allowable timestep 
can ensue. 

\item[$\bullet$] Each plasma species will contribute different levels of uncertainty, and 
a different scaling and performance overhead to the workflow. 

\item[$\bullet$] The ideal numerical algorithms for forming the Exascale edge plasma codes 
of the future will have at least the following properties: 

\begin{enumerate}
\item Accurate solution of hyperbolic problems. 
\item Accurate representation of velocity~(phase) space. 
\end{enumerate}

\item[$\bullet$] Options, which do not preclude consideration of others, have been tentatively 
identified for initial investigation as follows: High order spectral/hp finite 
elements.
\end{itemize}

The objective of this task is to produce an implementation of the model equations, 
geometry and boundary conditions described as System~2-5 entitled ``Spatially 
1-D multispecies plasma model'' of document~\cite{pappeqs} that: 

\begin{itemize}
\item[$\bullet$] Delivers an accurate and efficient solution together with information that 
will inform the roadmap to higher spatial order and increased complexity (full physics).
\item[$\bullet$] The bidder should indicate how they intend to assess ``accuracy" and ``efficiency",
together with performance and achievable problem size at Exascale as per Task 1.3.1.
\end{itemize}

\item[1.3.5]\textbf{Task 5: Delivery of System 2-6, 2-D plasma model incorporating velocity
space effects}

As stated in the Fusion Modelling System Science Plan:

\begin{itemize}
\item[$\bullet$] Work will initially focus upon coupling the turbulent plasma periphery to 
the surrounding neutral gas and partially ionized impurities that exist between 
the plasma and plasma facing components (co-design with work package FM-WP3). 

\item[$\bullet$] There may be an inadequate number of collisions for the species to equilibrate, 
leaving significant tails in the particle velocity distributions. 

\item[$\bullet$] The overall aim \ldots is to build a hierarchy of models that are capable of 
representing edge plasma behaviour to within a specific level of uncertainty\ldots 

\item[$\bullet$] The plasma fluid referent model may involve replacing kinetic models by 
moment-based or fluid models. 

\item[$\bullet$] The ideal numerical algorithms for forming the Exascale edge plasma codes 
of the future will have at least the following properties: 

\begin{enumerate}
\item Accurate* solution of hyperbolic problems. 
\item Accurate* representation of velocity~(phase) space. 
\item Accurate* representation of first wall geometry. 
\end{enumerate}
*See final bullet in this section.

\item[$\bullet$] Options, which do not preclude consideration of others, have been tentatively 
identified for initial investigation as follows: 

\begin{enumerate}
\item Spectral/hp high order finite elements. 
\item Nekmesh for mesh generation.
\end{enumerate}
\end{itemize}

The objective of this task is to produce an implementation of the nominal model 
equations, geometry and boundary conditions described as System~2-6 entitled 
``Spatially 2-D plasma model incorporating velocity space effects'' of the
document~\cite{pappeqs} that: 

\begin{itemize}
\item[$\bullet$] Delivers an accurate and efficient solution and information/guidance for 
the roadmap to the final full spatial 3-D code. 

\item[$\bullet$] The bidder should indicate how they intend to assess ``accuracy" and ``efficiency",
together with performance and achievable problem size at Exascale as per Task 1.3.1.
\end{itemize}

\end{itemize}

Building upon previous tasks within this call, the bidder is free to propose a 
more complex model above and beyond that described by the nominal equations for 
System 2-6 in~\cite{pappeqs} (including functionality from the preceding \Papp s where useful) 
as long as they deem this could be useful for increasing impact or quality or for 
de-risking or accelerating the \nep \   roadmap described at a high level in~\cite{sciplan} 
(to be agreed by UKAEA).

An option exists to exploit this \Papp \ to explore the adoption of DSL/code generator 
solutions, allowing the model to easily adapted (requiring co-design with others 
on the \nep \   team). The bidder is free to offer ownership of scope around DSL 
research and roadmap for \nep \   (as RO for developing the most important referent 
model within the project), or equally to indicate their willingness to work closely 
with UKAEA and others charged with this task. In either case, they should state 
their ideas and plans for how as a team, the project will eventually achieve a 
``separation of concerns'' within the final codebase. 

