This report comes as D3.3, a deliverable addressing the problem as to how the software
for project \nep\ should be structured in order to facilitate the development.
The original planning of the project envisaged that by this stage
there would have been time for more extensive interactions 
with the wider community, who will after all be largely responsible for the
provision of code.
However, such discussions have not been possible, owing
to the delays in the awarding grants and building the
community due to the COVID-19 pandemic.
Thus the material presented here
although based on the extensive work described in the previous milestone M3.3.x
reports~\cite{y1re331,y2re332,y2re333}, has to be regarded in part as a basis for further
negotiation with the community. Since it remains unclear how the software may be
best written so as to execute well on a range of different machine architectures -- and indeed
which architectures will successfully operate at the Exascale -- 
a choice of a parallelism abstraction layer (between eg.\ RAJA, Kokkos and SYCL)
continues to be  deferred~\cite[\S\,3.2]{y2re333}.
In the former case, details are expected to change in the light
of experience developing the \papp s, when hopefully the latter situation regarding the
choice of parallel library will have become clearer.
For the present, only  technical considerations to guide the selection
are presented. 

The survey of software patterns in the M3.3.x reports supports the contention that, in
the context of scientific software, patterns
should be viewed largely as a way of communicating the functionality of a module or subroutine.
Hence, as an important ultimate aim of the software is incorporation into a engineering design workflow
for a fusion reactor or its control systems, this report veers more to the model-based systems
engineering~(MBSE) viewpoint. A widely recommended language for
engineering design  is SysML\textsuperscript{TM}.~\cite{friedenthaletal}.
Conveniently SysML is underlain by the Unified Modelling
Language~(UML~\cite{omgumlwebsite}), already used in
refs~\cite{y2re332,y2re333} to describe software patterns, and
since \nep\ is a software project, more than adequate for present purposes.

Employing preferentially the UML nomenclature from UML~2,,
this report aims to provide a complete background against which to 
design the \nep\ software, complementing the M3.3.x reports.
\Sec{considers} discusses high-level constraints on the structure of software.
where the concept of division into packages and modules (which may represent libraries)  is promoted.
\Sec{scistruct} explores the implications of these constraints for \nep.
At the opposite extreme to \Sec{considers},
\Sec{lowlevel} describes the desirable contents for a single module,
and the last \Sec{parabstr} the smallest scale data structures relevant to selection
of an abstraction layer for machine parallelism.
\Sec{procio} discusses the question of what best to output when developing 
software for the Exascale.

%eg.\ Douglass's book~\cite{douglass}.
