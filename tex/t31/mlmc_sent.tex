An alternative approach to producing the PCE is to sample at a set of $N$~random
points~$\theta_j, j=1\ldots N$, so that
$n$ is a distribution in probability space, viz.
\begin{equation} \label{eq:dfnexpn}
n(x,t,\theta) = \sum_{j=0}^{N} n_{sj}(x,t)\delta(\theta-\theta_j)
\end{equation}
and then to equate with \Eq{Hermite_expansion} in the weak sense, ie.
\begin{equation} \label{eq:hermweak}
\langle\sum_{j=0}^{N} n_{sj}(x,t)\delta(\theta-\theta_j), H_k(\theta)\rangle = 
\langle\sum_{j=0}^{N} n_{j}(x,t) H_j (\theta), H_k(\theta)\rangle 
\end{equation}
where $\langle,\cdot \rangle$ denotes the inner product indicated by \Eq{}, so that
\begin{equation} \label{eq:hermres}
n_k(x,t) = \frac{1}{2^nn!\sqrt{\pi}}\sum_{j=0}^{N} n_{sj}(x,t)H_k(\theta_j) e^{-\theta_j^2}
\end{equation}

\subsection{MLMC}\label{sec:mlmc}
%MFMA - see Slide 7 of Sa18Spar-ppt
Multi-level Monte-Carlo~(MLMC) is a more efficient way of estimating averages
and other statistics by means of random or Monte-Carlo~(MC) sampling. 
MLMC is usually associated with the name of Mike~Giles~\cite{Gi16Mult}.
Suppose that a model or surrogate problem may be solved with a range of
spatial and/or temporal resolutions. The basic idea is that sample solutions
computed with different levels of resolution may be combined to produce means
and other statistics 
(giving estimates of uncertainty) as accurately as though all solutions had been
obtained using the finest level.

In more detail, for the example of a steady solution obtained with
different spatial levels of resolution~\cite{Mi13Mult}, the MLMC algorithm is
\begin{enumerate}
\item Construct a hierarchy of meshes, spacings~$h_l$, where $l=0$ is the coarsest level
and $l=L$ is the finest.
\item For each~$l$, draw a level-dependent number~$N_l$ of samples from the parameters/fields
to be sampled, eg.\ 
${\bf U}$ denoting a single solution depending on spatial variables alone.
\item Solve the model or surrogate problem $N_l$ times at each level, giving an
ensemble of solutions~${\bf U}^k_l,\;k=1,\ldots N_l$,
\item Estimate the mean (other moments of the distribution follow similarly) as
\begin{equation}
{\mathbb E}^L[{\bf U}]= \sum_{l=0}^L {\mathbb E}_l[{\bf U}_l-{\bf U}_{l-1}]
\end{equation}
using simple averages~${\mathbb E}_l$ with ${\bf U}_{-1}={\bf 0}$
\end{enumerate}
Large gains occur when
$N_l=N_0 4^{lp}$ where the solution scheme is of order~$p\leq(d+1)/2$.

