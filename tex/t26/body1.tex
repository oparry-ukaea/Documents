\section{{\nep \  Meeting: 17 June 2021 11.00-12.00 BST}}

\emph{Present}

\begin{itemize}
\item{Wayne Arter, UKAEA}
\item{Ben Dudson, University of York}
\item{Ed Threlfall, UKAEA}
\end{itemize}

\section{Minutes}

{\it This was a one-on-one meeting between UKAEA and Ben Dudson, holder of the 
grant T/NA083/20, to discuss progress.}

WA explained that this meeting was the basis for a progress report to the Met 
Office, serving to identify the latest intelligence on promising 
research directions.  Another reason for this particular session was to discuss 
future plans in light of staff movements at York, which was discussed first.

\subsection{BD's relocation and consequences}\label{sec:reloc}

BD provided an update on the structure of York's future role in \nep \  : David 
Dickinson is currently spending 10\% of his time on the project, to rise to 
20\%; he is taking on tasks linked to Oxford (ie.\ 1-D kinetics).  BD gave an 
update on Ed Higgins - BD is liaising with Emma Barnes, the effective `head' of 
RSEs at York, re EH taking over from Peter Hill (n.b. EH now plans to stay at 
York).  EH has background in material science but has worked on automated 
testing framework, community coordination etc (typical RSE activities) but 
lacks the fusion experience of PH.  WA expressed some concern at only getting 
20\% of DD's time - BD said the latter had other commitments inc. doctoral 
training (CDT) and STEP.  WA said he would have to recommend a deceleration of 
affected parts of the \nep \   schedule.  BD agreed that some aspects should be 
prioritized eg.\ kinetics (DD's main interest) and conceded that Task 5 
(complicated 2-D fluid model) would have to be assigned lower priority.  WA said 
the latter was something of a pity as collaboration with TSVV would have been 
possible and a superior version of that code would have been an asset.

WA asked about the position re BD working on \nep \   after taking up his new 
post at Lawrence Livermore National Laboratory~(LLNL).  BD made it clear 
that his terms preclude accepting any payment except through the US DoE, so 
would need to propose formally to the DoE in order to work on \nep \  ; he 
thought there might be an acceptable case for this.  WA cited his own private 
sector experience and also mentioned that UKAEA have ongoing programmes with 
some US labs, especially for fusion work (Rob Akers' area).  BD had in the past 
liaised with Rob Akers and Fulvio Militello re setting up a UKAEA-LLNL
memorandum of understanding (MoU), with an agreement to develop software for fusion activities.

BD's future employer is keen for him to continue \F{BOUT++}
development - WA asked what the key aims were and whether this included 
extensions to higher-order methods.  BD replied that the main interest is in 
incorporating additional physics (he has in the past tried adding a 
fourth-order accurate finite-difference scheme but it was fragile on a 
non-uniform grid).

\subsection{Main Discussion}\label{sec:maindisc}

WA asked whether BD was still happy to implement the multispecies plasma model
(Task~4 for BOUT++, System 2-5 in ref~\cite{pappeqs}); BD 
answered that much of this is already underway under the auspices of STEP (BD 
shared a link to Hermes-3~\cite{hermes3website}  via the chat); the concept is a modular 
multispecies model.  WA asked whether the design could handle adding different 
species, to which BD replied yes and offered the manual as a source of further 
information~\cite{hermes-3strucwebsite}.  WA asked whether the design had a 
generic species with attributes such as mass and charge.  BD gave an 
explanation of two designs he had tried.  1) In the existing (1-D) SD1D code he 
had introduced a species class and also reactivity functions which acted to 
provide sources/sinks - this worked well in 1-D but was cumbersome in 2-D/3-D. 
2) A design with a state object was tried - reactions etc. modify this 
state, which passes down the task chain; this approach was more flexible but it 
lacked `checks and balances', also any task could modify the state, leading to 
ordering and calling convention issues and the possibility of using unset values.
In summary, the latter scheme was not fully robust.  WA agreed that design 2) would 
involve a lot of work.  BD promised to write up both designs.  WA made the 
point that design 2) has similarity to the mechanics of computer games and BD 
admitted that the ENTT framework used by \F{Minecraft} had been an inspiration; WA 
mentioned in this context that \F{Maya} is well-documented, see book by Gould~\cite{gould},
although of course \F{Maya} now as an Autodesk product is commercial software.
BD indicated that he has specifically considered  ENTT 
and provided a github link~\cite{enttwebsite}.  Regarding actual equations, BD said he would look at 
the terms in the equation system 2-5 of the call and check that they can be added.  WA said 
documentation would be very important for future work.

The discussion moved onto the area of progress.  WA mentioned work done by Will 
Saunders on higher-order methods for 1-D fluids in \F{Firedrake}: boundary 
conditions were being investigated but WA admitted the grid used was currently 
`generous' in number of FE nodes- this work is progressing.  BD agreed this was promising.  WA said 
UKAEA is leveraging WS's existing \F{Firedrake} experience, though he has a 
background in molecular dynamics and not plasmas.  It was suggested that some 
of BD's workload could be taken up by WS. However, there is still a need for a 
1-D multispecies code, for which UKAEA needs evidence that it works with 
higher-order methods.  WA said the intention is for Sarah Newton to help guide 
WS toward solutions of more realistic problems to explore the limits of \F{Firedrake},
although the recently received report~\cite{2047356-TN-04} on test problems
had now also been passed on to WS.
WA said the current lesson from \F{Nektar++} simulations  is that if 
there is a problem, decrease the mesh size,
possibly by whole orders of magnitude.  BD asked whether the 
\F{Firedrake} DSL was still a candidate for becoming a common language for 
the various frameworks involved by \nep.  WA mentioned the opinion of Steven 
Wright that \F{Firedrake}'s generated C code is opaque and not readily 
maintainable in itself (BD called it a one-way translation).  WA said \nep\ 
potentially could fund software that was better in this regard: UFL is not an 
acceptable option for non-FEM-conversant physicists (eg.\ because of the  need to understand 
weak form).  WA mentioned the issue for FEM of spatially-dependent parameters such as the
thermal conductivity, and the many different terms appearing in eg.\ Zhdanov~\cite{zhdanov}.
The standard solution is to assign nodal (point) values; BD confirmed that \F{BOUT++} 
does something similar eg.\ at cell boundaries, but made the point that proofs of 
convergence become equivocal in the presence of such complexities.  ET 
mentioned his own experience with discontinuous Galerkin was for physical 
parameters to be elementwise constant.  WA said this might be acceptable for 
slowly-varying quantities, but BD brought up the fact that, in plasma physics, 
some quantities vary as $T^{\frac{5}{2}}$ which implied rapid spatial 
variation.  WA said the opinion of Spencer Sherwin should be sought.

WA then asked BD specifically if any `showstopper' problems had come to light; BD 
replied that he was not aware of such; however, David Moxey had identified the 
need to make significant changes to \F{Nektar++} in order to incorporate 
velocity-space effects (eg.\ a different basis set).  WA said that
he had hoped to be further advanced by now, notably it was still unclear whether 
kinetics problems should be solved via particle methods or 5-D/6-D version of 
\F{Nektar++}.  WA asked BD to confirm that \F{BOUT++} still does only 
finite-difference - BD confirmed.  WA asked about conservative vs. 
non-conservative finite-difference - BD's opinion was that the choice sometimes 
makes rather little difference, being of greatest import in a closed system; in 
the scrape-off layer, it may not be such an issue due to the large throughflow 
of energy.  WA mentioned concerns about mass conservation; BD said this was 
mostly important for high-recycling systems where absolute fluxes are large 
compared to sources.  WA said charge conservation was also important eg.\ in 
PIC schemes.  It was agreed that any invariants of a code should be explicitly 
checked; and also anything physically conserved but not constrained as constant 
should also be checked.  (WA said that he understood that the ideal scheme conserving mass,
energy, and momentum in the absence of dissipation was very hard to realise in practice.)
WA asked about these sorts of checks in \F{BOUT++}.
BD said there were routines for mass conservation checking but that 
energy conservation was more complicated as the operators for eg.\ $\lvert 
\nabla u \rvert^2$ need to be implemented identically in post-processing code.
Post-processing in 
\F{BOUT++} is via a Python back-end.  WA mentioned that another problem of 
finite-differences is that refinement to ensure local energy conservation means 
global refinement due to the nature of the grid, but  BD had little to say on 
specifically mesh refinement, though there is some non-dynamical boundary 
refinement in \F{BOUT++}.  WA agreed meshing questions were David Moxey's 
area (BD's main role lies in the physics, though he is currently learning FEM).
  
WA asked whether any `showstoppers' has arisen in Hermes-3 - BD answered that 
there were no new issues, just slowness due to equations that already were 
known to be extremely stiff.  WA asked about coupling - BD replied that the 
system was tightly-coupled with all species solved simultaneously.  WA said it 
would be a good problem for a preconditioner and BD agreed, stating also that 
the choice of time-integrator was key - he had been searching for quick 
and robust schemes eg.\ backward Euler.  WA said there might be \nep\ resource to 
investigate time-stepping; BD mentioned SUNDIALS, which is the default choice 
for \F{BOUT++}.  WA asked for BD's opinion on time-steppers, given that 
higher-order Runge-Kutta (RK) seems currently fashionable.  BD replied that he 
has not tried implicit RK but that SUNDIALS is generally hard to beat.  WA 
favoured investigating high-order implicit RK; BD said that he had heard good 
things in that area from Met Office.  WA said that it was natural to use RK for 
discontinuous Galerkin, which is a kind of RK in spatial coordinate in that 
each completed time advance requires only values within the timestep.

WA summarized that the main emphasis was in the physics and that no new 
problems had so far been revealed (especially issues that scale badly), leaving 
only known problems such as the~$N^2$ complexity with number of species~$N$.  
BD agreed, reminding that the size of the state vector also scaled 
with the grid size. 
WA said the problem of multispecies coupling was one area 
where the power of Exascale HPC appeared to be absolutely necessary (though 
obviously Exascale would also help in many other areas).  BD said the problem should be 
decoupled if possible.
WA responded that local matrix inverses can be stored
but he agreed that they were unlikely to remain usefully constant.

The conversation turned to uncertainty quantification (UQ), with WA citing 
mentions of intrusive UQ earlier in the project (two years ago).  The current 
position, following involvement from the VECMA community, was to favour 
non-intrusive UQ, which is easy to parallelize.  BD mentioned 
auto-differentiation and its similarity to the kind of sensitivity analyses 
done in UQ.  WA opined that intrusive UQ is not necessarily difficult to implement
but that it involves 
significant amounts of labour.  WA asked BD to postpone if not cancel work on intrusive UQ,
adding that non-intrusive UQ was more in line with 
the \nep \   philosophy of separation of concerns. BD agreed, although indicating that
he felt it should ultimately be pursued.  WA emphasized the need to 
keep the UQ grant holders busy (`fed' with work) as those contracts end soon 
and the grant holders ideally should focus on fusion as a use case.

WA asked whether BD had any further matters to raise.  BD answered largely in 
the negative, but again mentioned the time-stepping issue; global 
communications and small features (stability limit) constrain the 
time-stepping; the possibility of non-local or asynchronous time-stepping was 
aired.  WA made the point that time is a `additional coordinate' in most of \nep \   
numerics; and he mentioned a recent UKAEA Advanced Computing talk on 
parallel-in-time evolution (by Debasmita Samaddar of UKAEA) which applied 
parareal time-stepping to SOLPS - showing significant improvements in wall-clock time
but accompanied by a similarly increased demand for data storage. 

