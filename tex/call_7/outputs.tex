\begin{enumerate}
\item Investigation of at least two different approaches to
matrix preconditioning and associated matrix structures
that show promise of suitability for HPC / Exascale implementation particularly
in the context of a spectral element code for plasma modelling,
treatable as a set of interacting compressible and incompressible fluids with complicated source terms.

\item Potentially suitable approaches will be evaluated by means of  
literature and code surveys and consultation
with UKRI and industry experts.
Relatively small development tasks may initially be undertaken to  
test candidate methods for 
accuracy, stability and HPC scalability potential.
At minimum, evaluation should produce an objective assessment as to
the appropriateness of each matrix preconditioning technique as set out in the 
above objectives section. For each too should be similarly assessed
whether an approach is suitable for:
\begin{itemize}
\item Immediate implementation in \nep\ defined in the maturing project roadmap
and through co-design with project partners initially in the Y1-2 \papp s.
\item Recognition by the \nep\ design so that deferred implementation is facilitated.
\item Implementation only in a restricted class of machine architectures, thereby
steering architecture choices and priorities for future infrastructure.
\end{itemize}

\item Comparison between use in close-coupled and loose-coupled solution.

\item The bidder could consider producing \papp s, perhaps as  
part of a response to other elements of this \nep\ call, to  
demonstrate key features of preferred approaches.

\end{enumerate}

%\input{../commitment0}
