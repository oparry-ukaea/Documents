\begin{itemize}
\item This task focuses upon the suitability of available numerical 
algorithms (or the development of new algorithms) for matrix preconditioning in the
context of Exascale targeted plasma modelling. As stated in the Fusion Modelling System
Science Plan, elements of this work package FM-WP1 together with FM-WP4 will also
surround the ``coupling technology'' that will be required to connect the 
edge/pedestal region of the plasma (addressed
by FM-WP2) and the neutral gas/impurity model (FM-WP3).

\item The ideal numerical algorithms for forming the Exascale edge plasma codes of 
the future will have preferably at least the following properties:
\begin{itemize}
\item[P1] Accurate solution of hyperbolic problems.
\item[P2] Ability to deliver efficient and accurate solutions of corresponding 
elliptic problems.
\item[P3] Accurate modelling of highly anisotropic dynamics. 
\item[P4] Accurate representation of first wall geometry (face normals to
within~$0.1^{0}$), and correspondingly of complex magnetic field geometries.
\item[P5] Accurate representation of velocity (phase) space.
\item[P6] Preservation of conservation properties of the underlying equations.
\item[P7] Scalability to likely Exascale architectures:
\begin{itemize}
\item[a] interaction between models of different dimensionality,
\item[b] interaction between particle and fluid models,
\item[c] dynamic construction of surrogates.
\end{itemize}
\item[P8] Performance portability to allow rapid deployment upon emerging hardware.
\end{itemize}

\item It is unlikely that any algorithm will have all the above, and part of the exercise
will be to rank the importance of these properties.

\item 
The availability of appropriate matrix preconditioning may influence the choice
of numerical scheme, eg.\ ref~\cite{Ya16ToCG} and therefore have a profound impact upon
almost all areas of the project.

\item Options have been tentatively 
identified for initial investigation as follows:
\begin{enumerate}
\item Spectral/hp element, combined with Discontinuous Galerkin, to meet P1,P3
\item Multigrid methods for P2
\item For Exascale (P7):
\begin{enumerate}
\item matrix-based approaches, hierarchical geometric structures,
\item kinetic enslavement~\cite{taitano}, multi-index Monte-Carlo methods
\end{enumerate}
The above items require solution of linear systems of equations $A{\bf x} = {\bf b}$
for field values~${\bf x}$, hence
are expected to benefit from appropriate matrix preconditioning, ie.\
algorithms which are mathematically equivalent to forming the
matrix product $M=C^{-1}A$ so that $M{\bf x}=C^{-1}{\bf b}$ is easier to solve,
typically because iteration proceeds faster as $M$ is closer to a diagonal matrix.
It may help to note that in the comoving system of coordinates (Lagrangian formulation), 
transport operators reduce to the identity.

\end{enumerate}
\end{itemize}
