\subsection{Software design patterns for Call 6 Proxyapps}\label{sec:call6}

%Description of proxyapps

The Contract Ref.\ T/NA083/20 specifies the creation of five proxyapps to
mirror the development of the referent fluid model.
This model will capture the physics in the tokamak edge, including the
separatrix but away from the core and from the metal surfaces of the device.
It is to include the multi-physics phenomena relevant to the tokamak exhaust
(bulk plasma species, impurities and neutrals).  
The code itself must be performant and scalable to Exascale, easy to deploy
upon different architectures, and actionable (i.e.\ include treatment of
uncertainty quantification, UQ).  
Moreover it must couple efficiently to the neutral gas and impurity model, and
the 5D gyro-averaged model developed in other work packages.

The five proxyapps are largely independent, focussing on separate aspects, namely:
\begin{enumerate}
	\item 2D elliptic solver in complex geometry
	\item 1D simplified fluid model with UQ and realistic boundary conditions
	\item 1D model incorporating velocity space effects
	\item 1D multispecies plasma model
	\item 2D model incorporating velocity space effects
\end{enumerate}

%Context of proxyapp

The two proxyapps focussing on velocity space effects (proxyapps 3 and 5) will draw on work on
kinetic solvers, and will be written in a high-level language such as Python or Julia.
For each of the remaining proxyapps, two versions will be developed for
comparison, one in BOUT++ \cite{BOUT++} and one in Nektar++.
As mentioned above, Nektar++ is a framework for solving PDEs with spectral/hp, which will be extended as part of Call 1 to include terms relevant for plasma physics.
In contrast, BOUT++ is a framework for solving plasma and fluid-like equations in curvilinear coordinates using finite-difference approaches,
but lacks the complex meshing and finite element approaches of Nektar++.

Proxyapps 2 and 4 above investigate the feasibility of including new physics
(namely realistic sheath boundary conditions and multiple species). 
For this, new terms will be added to the existing BOUT++ physics modules, SD1D
and Hermes-3.
Owing to BOUT++'s domain specific language, these additions will be relatively
straightforward (from a software development perspective).

The remaining proxyapp (number 1) will be implemented by coupling BOUT++ to
elliptic solvers provided by PETSc and Hypre.
BOUT++ uses the strategy pattern (see section \ref{sec:strategy} below) to implement different elliptic solvers.
BOUT++ also makes use of the template method (section \ref{sec:template}) and abstract factories (section \ref{sec:abstract_factory}).

