This report has the goal of consolidating earlier work regarding software
design patterns \cite{y2re332} by attempting an initial identification of
specific patterns to be applied in the forthcoming \nep\ \papp s, and also
summarizing some other general design concerns.

Section \ref{sec:proxypat} identifies patterns relevant to
specific core \nep\ \papp s.
It is noted that two of these \papp s are to be based on pre-existing
object-oriented (C++) frameworks and the task is made easier by the explicit
identification of some of the relevant patterns in those frameworks'
accompanying documentation.
Those are the \papp s that are to treat continuous fields; an additional part
describes one of the \papp s designed to treat particles, which at the time 
of writing is intended to be based on a procedural code written in Fortran.  
Further insight into the use of design patterns in Fortran, and more generally the
construction of complex modular frameworks, can be 
gleaned from the study of existing codes; to this purpose, a discussion of 
the \F{SMARDDA} simulation framework is included.
Following sketches of the \papp s, a shortlist of relevant design patterns is
given.

Section \ref{sec:overarchpat} presents some overarching concerns
relevant to the initial \papp\ ecosystem as a whole, viewed here as a
fledgling multiphysics ecosystem and keeping in sight the likely structure of
exascale target hardware.
The first annex \ref{sec:gamestruct} demonstrates one tangentially relevant approach
to handling a complex software ecosystem.
As in earlier work, a general dearth of published material treating patterns in
scientific programming is noted, indicating that room is left for more definite
strategies for the overall \nep \ framework, once the initial implementations
of the \papp s become available.

A second  annex \ref{sec:proxyapp_standards} contains a precis of a set of standards
for scientific \papp s and a framework for benchmarking the same proposed by
the US Department of Energy's Exascale Computing Project.
