This report has two aims as prefigured in the report~\cite{y1re311}. The central \Sec{taskwork} mainly focusses on
design patterns to be used in \nep, but also pursues the prototyping process.
It begins in \Sec{genpat} by outlining in some detail the design patterns that historically have proven useful in large software engineering projects,
starting with general software design before specialising in \Sec{scipat} to scientific programming and multiscale physics simulation.  
In \Sec{proto}, consideration is given to the past and current contexts of software development processes: it is worth noting that,
today, the HPC landscape is in something of a state of flux with the rise of heterogeneous architectures and the
corresponding coding tools. There is a final summary \Sec{summ}.

The original design pattern concept is credited by Sommerville (\cite{sommerville}, p.209) to architect and design theorist
Christopher Alexander, who in his 1977~book {\it A Pattern Language: Towns, Buildings, Construction}~\cite{christopheralexander}
presented a compendium of {\it `certain common patterns of building design that are inherently pleasing and effective'}.
His text outlines some~253 such patterns, described collectively as a `design lexicon' or even an entirely new tongue:
to quote Alexander himself {\it `All 253 patterns together form a language.'}.

A similar approach was applied to object-oriented software architecture in the `90s, resulting in the publication
of {\it Design Patterns: Elements of Reuseable Object-Oriented Software} \cite{gammahelmjohnsonvlissides}, the
authors of which have become known as the Gang of Four~(Go4, also abbreviated GoF).
This book proposed 23 design patterns that are classified according to a trinity of themes: creational patterns,
concerning object generation; structural patterns, to do with classes and composition, and behavioural patterns,
dealing with interactions between objects.
These will be discussed with a view of how germane each is to project \nep \ (it is also relevant that,
during the time since the publication of \cite{gammahelmjohnsonvlissides}, some of the original patterns
have been elevated to the status of permanent language features in some object-oriented languages).
Note that there is now a fourth class of design patterns (as exhibited in the Wikipedia article
\cite{softwarepatternwiki}) concerning concurrency patterns.
%The Wikipedia article on patterns includes 
%Resource acquisition is initialization (RAII), which is perhaps generally regarded more as a
%concept, an important technique for avoiding memory leaks, whereby resource is acquired in the
%constructor of an object and released in its destructor~\cite[\S\,4.2.2]{stroustroptour}.

Aside from these general concerns, there also exist patterns which apply specifically to {\it scientific} software.  
As detailed in \Sec{rouson}, the textbook by Rouson, Xia and Xu~\cite{rousonxiaxu} develops most of
a framework explicitly targeted to multiphysics workflows
for Exascale HPC, starting from a foundation of object-oriented principles.  
Their text presents a viewpoint on the most useful Go4 design patterns in the context of scientific programming
and also offers some novel patterns tailored specifically to this field.  

Consideration is given in \Sec{compat} to the ComPat project \cite{compatwebsite}, which, being a framework and software
suite to study multiscale fusion plasmas on HPC systems, constitutes a further specialisation in the direction
of project \nep.  
Design patterns here are used to provide separation of concerns, with the full physics being represented by
an interchangeable set of submodels in a coupling framework.  
Attention is given to the preservation of good scaling to Exascale HPC in such a framework.
%vecmatk
%A description is given of a library for coupling the components of a multiscale application ...
\Sec{vecmatk} gives more details of the Verified Exascale Computing for Multiscale Applications~(VECMA) project~\cite{vecmawebsite}
which essentially represents a continuation of ComPat, using updated versions of the same components
eg.\ the multiscale coupling framework MUSCLE.  The associated VECMA toolkit~\cite{vecmatkwebsite}
provides a platform for VVUQ.  Although not a pattern in the strict sense, this suite is of interest
as it employs several more fundamental patterns (for example the Go4 {\it Adapter}) in a framework
which includes the management of a set of interrelated jobs on either a cluster or an HPC machine,
and the provision of graphical and Python interfaces.

%The Wikipedia article on patterns includes 
%Resource acquisition is initialization (RAII), which is perhaps generally regarded more as a
%concept, an important technique for avoiding memory leaks, whereby resource is acquired in the
%constructor of an object and released in its destructor~\cite[\S\,4.2.2]{stroustroptour}.

In software engineering, prototyping is normally described as
a quick way to produce something that potential customers can explore, primarily with a view to
improving the user-interface. Only Booch~\cite[\S\,8.1]{booch} admits that it may help the developer
understand the technical aspects of the problem better, and this work predates design patterns.
As indicated in the previous report~\cite{y1re331}, little formal description has been found of the role
of prototyping in scientific computing. However the highlighted `idea' paper of  Dubey and McInnes~\cite{Du16Idea}
does include both these applications and is further discussed in \Sec{dubey}.

