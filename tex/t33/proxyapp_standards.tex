The US Department of Energy's Exascale Computing Project (ECP) \cite{ECP_website} has curated a suite of proxyapps \cite{ECP_proxyapp_website}
to benchmark performance for exascale.
Along with this, ECP have developed guidance for quality standards and best practices that proxyapps must meet in order to merit inclusion in the suite.
Each term in the guidance is designated ether required or recommended,
with the expectation that the guidance becomes stricter over time, with
recommendations becoming requirements.

These standards are available online \cite{ECP_proxyapp_standards_website},
and are concise and readable.
The standards have also been adopted by the ExCALIBUR Benchmarking working
group as a basis for the rules to use within ExCALIBUR.
We therefore give a condensed description.

The guidelines require that a stable release of the source code must be
publicly available.
This perhaps presents a problem for nuclear software, as private repositories
or any other form of limited access are not deemed acceptable.

Proxyapps should be limited in size to $\lesssim 10,000$ lines of code, aiming
to encapsulate repeated code or pass it off to libraries.
At the same time, the proxyapp should be self-contained, with minimal
dependencies handled through a package manager (Spack is recommended).

The guidelines recommend three running modes.
Firstly, a (required) testing mode to verify correct compilation and kernel
performance (with execution time recommended to be on the order of seconds).
This mode should also include an ``automatable'' regression test to check the
final output against a known reference result.
Two further production modes are recommended to allow tracking of weak and
strong scaling for 
(1) current workloads; and 
(2) anticipated exascale workloads.

The guidelines place a strong emphasis on documentation.
There are basic requirements for a general description of the proxyapp, 
its algorithm, and what aspect of the parent app it imitates, 
and for documentation of the build and testing process.
In addition, it is required that documentation should describe where the
proxyapp is -- and is not -- representative of its parent.
It is also recommended that documentation should detail which ``simple''
optimizations have \emph{not} been applied to the proxyapp, as doing so would
not feasible in the parent.

Finally, the proxyapp must include a ``figure of merit'' that can be used to
compare performance under different conditions. 
This might be based on the system being modelled (e.g.\ accuracy of a solution)
or on what the proxyapp is stressing (e.g., run time, computational power or
memory). 
In any case, the figure of merit should be representative of the performance of
the parent application.
If possible, it is recommended that the proxyapp provides a performance model
of expected outcomes of the test in different scenarios.


