Acronyms used in t31 work to go here from file t31acro.txt.

One of my favourite statistical techniques is \gls{ANOVA}.
The full name and acronym are printed on first usage, but thereafter the same command prints ``\gls{ANOVA}''.
\glsreset{ANOVA}
Calling \verb+\glsreset{ANOVA}+ means the full name is printed next time we write \gls{ANOVA}.
\glsreset{ANOVA}
\Gls{ANOVA} is capitalized by using \verb+\Gls{ANOVA}+, while the plural is provided by \verb+\glspl{ANOVA}+, giving \glspl{ANOVA} or, after resetting,
\glsreset{ANOVA}
\glspl{ANOVA}.
The (possibly incorrect) ``analyses of variance'' and (definitely incorrect) ``ANOVAE'' plurals are produced by the `longplural' and `shortplural' optional arguments to \verb+\newacronym+.
Without these, the defaults would be ``analysis of variances'' and ``ANOVAs''.

By default, only acronyms that are referenced are included in the glossary.
This makes it difficult for the glossary to be a separate document from the reports (and it certainly breaks the page number cross-referencing).
One work-around for this would be to use the command \verb+\glsaddall+ to include all terms in the glossary file, and then turn off the page numbers by including the argument \verb+[nonumberlist]+ in the \verb+usepackage+ command.

\glsaddall



\printglossaries

%\begin{longtable}[h]
%\begin{center}
\begin{longtable}{|p{4.0cm}|p{12.0cm}|}
\hline
ACM & \\
ADC & \\
AMR & \\
AMReX & \\
AND & \\
ANL & Argonne National Laboratory \\
ANN & Artificial neural network \\
ANOVA & Analysis of Variance \\
API & Application Programming Interface \\
ASQ & Adaptive sparse quadrature \\
ATS & Advanced Terrestrial Simulator, previously Arctic Terrestrial Simulator \\
AUXVAL & \\
BD & \\
BG/L & \\
% BOUT++ & proper noun not acronym
BOX & \\
BP & \\
CabanaMD & \\
ECP-copa & \\
CCFE & Culham Centre for Fusion Energy \\
CD & \\
CEA  & \\
CESM & \\
CI & \\
COGENT & \\
COMPAT & \\
COSMO & \\
COSSAN & \\
CPP & \\
CPU & Central Processing Unit \\
CRC & \\
CS & Compressed sensing \\
CSE & \\
CSMP & \\
CTO & \\
CUDA & Compute Unified Device Architecture \\
CWIPI & \\
DAG & Direct Acyclic Graph \\
DAKOTA & \\
DDA & \\
DFT & Discrete Fourier Transform \\
DOE & Department of Energy \\
DOI & Digital Object Identifier \\
DSL & Domain-Specific Language \\
ECP & \\
EIRENE & \\
ESI & \\
ESMF & \\
ETS & \\
EU & European Union \\
FACETS & \\
FCI & Flux-Coordinate Independent (methods) \\
% FELTOR proper noun
FEM & Finite Element Method \\
FEniCS & \\
FFT & Fast Fourier Transform \\
FFTW & Fastest Fourier Transform in the West (library) \\
FLASH & \\
% FORTRAN
% FORTRAN~77
GA & General Atomics \\
GBD & \\
GBS & \\
GDB & \\
% GENE proper noun?
GMRES & Generalized Minimal Residual method \\
GNU & GNU's Not Unix! \\
gPC & Generalised polynomial chaos \\
GPU & Graphics Processing Unit \\
% GRILLIX proper noun?
GSA & Global sensitivity analysis \\
GUI & Graphical User Interface \\
HDF5 & Hierarchical Data Format (version 5) \\
HLA & \\
HPC & High Performance Computing \\
HVAR & \\
% IBM & International Business Machines Corp., but really known as IBM \\
ICON & \\
IEEE & \\
IMAS & \\
IMEX & Implicit-Explicit Methods \\
IO & \\
% ITER *not* International Thermonuclear Experimental Reactor
ITG & Ion Temperature Gradient \\
ITM & Ion Tearing Mode \\
ITPA & \\
JET & Joint European Torus \\
JOREK & \\
LAMMPS & \\
LANL & Los Alamos National Laboratory \\
LASH & \\
LASSO & Least Absolute Shrinkage and Selection Operator \\
LGPL & GNU Lesser General Public License \\
LHSamp & \\
LLNL & Lawrence Livermore National Laboratory \\
MAP & Maximium A Posteriori \\
MC & Monte Carlo (methods) \\
MCMC & \\
MCT & \\
MESAGE & \\
MFMA & \\
MFMC & Multi-fidelity Monte Carlo \\
MHD & Magnetohydrodynamics \\
MIMC & \\
MIT & Massachusetts Institute of Technology \\
ML & Machine Learning \\
MLMC & Multi-level Monte Carlo \\
MLMF & \\
MMF & \\
MOOSE & \\
MOR & \\
MPI & Message Passing Interface \\
MSSC & \\
MUMPS & \\
MUSCLE~3 & \\
% NAG trades a NAG
NEMO & \\
NEPTUNE & \\
NUCODE & \\
OASIS & \\
OASIS4 & \\
OLYMPUS & \\
OMFIT & \\
% OpenMP & Open Multi-Processing, but not really an acronym
OU & Oxford University \\
OUU & Optimisation under uncertainty \\
OUUWA & \\
PASTIX & \\
PC & Polynomial chaos \\
PCE & Polynomial chaos expansion \\
PDE & Partial Differential Equation \\ 
% PETSc & Portable, Extensible Toolkit for Scientific Computation, but again, really it's PETSc
PFC & \\
PGD & \\
PICPIF & \\
POOMA & \\
PP20 & \\
PRESET & \\
QA & \\
QCG  & \\
QMC & Quasi-Monte-Carlo \\
QoI & Quantity of interest \\
RKF23 & \\
ROM & Reduced-order model \\
RNG & \\
SAMRAI & \\
% SD1D name of code
SIAM & Society for Industrial and Applied Mathematics \\
% SLEPc & Scalable Library for Eigenvalue Problem Computations, but really a name
SLSQT & Sequential Least-Squares' Thresholding \\
SMARDDA & \\
SMART & \\
SMITER & \\
SMwiki & \\
SNOWPAC & Stochastic Nonlinear Optimisation with Path-Augmented Constraints (software package) \\
SOL & Scrape-Off Layer \\
SOLEDGE & \\
SOLPS & \\
SRS & \\
STARWALL & \\
STIXGeneral & \\
% STORM
STRUMPACK & \\
% SUNDIALS
SVD & Singular value decomposition \\
SVM & \\
TAE & \\
TM & \\
TOKAM & \\
TOKAM3X & \\
TOMS & \\
TRIMEG & \\
TUM & \\
UK & United Kingdom \\
UKAEA & United Kingdom Atomic Energy Authority \\
UKRI & United Kingdom Research and Innovation \\
UQ & Uncertainty quantification \\
US & \\
USA & \\
UTF-8 & \\
VDE & \\
VECMAtk & \\
VORPAL & \\
XGC1 & \\
XML & \\
XMSF & \\ \hline
\end{longtable}
%\end{center}
%\end{table}

\newglossaryentry{computer}
{
  name=computer,
  description={is a programmable machine that receives input,
               stores and manipulates data, and provides
               output in a useful format}
}

\newglossaryentry{pi}
{
  name={\ensuremath{\pi}},
  description={ratio of circumference of circle to its
               diameter},
  sort=pi
}



\Gls{pi}
\Gls{fpsLabel}

\begin{table}[h]
\textbf{\textsf{TABLE OF SYMBOLS}
Scan of Brunton and Kutz acronyms table converted to text
}
\begin{center}
\begin{tabular}{|p{4.0cm}|p{12.0cm}|}
\hline
\textbf{\textsf{Acronym}}
& \textbf{\textsf{Description}} \\
\hline
ADM  & Alternating directions method \\
AIC  & Akaike information criterion \\
ALM  & Augmented Lagrange multiplier \\
ANN  & Artificial neural network \\
ARMA  & Autoregressive moving averaoe \\
ARMAX  & Autoregressive moving average with exogenous input \\
BIC  & Bayesian information criterion \\
BIM  & Empirical interpolation method \\
BPOD  & Balanced proper orthogonal decomposition \\
CCA  & Canonical correlation analysis \\
CFD  & Computational fluid dynamics \\
CNN  & Convolutional neural network \\
CoSaMP  & Compressive sampling matching pursuit \\
CWT  & Continuous wavelet transform \\
DCT  & Discrete cosine transform \\
DEIM  & Discrete empirical interpolation method \\
DFT  & Discrete Fourier fransform \\
DiMDc  & Dynamic mode decomposition with control \\
DL  & Deep learning \\
DMD  & Dynamic mode decomposition \\
DMDc  & Dynamic mode decomposition with control \\
DNS  & Direct numerical simulation \\
DWT  & Discrete wavelet transform \\
ECOG  & Electrocorticography \\
eDMD  & Extended DMD \\
EM  & Expectation maximization \\
EOF  & Empirical orthogonal functions \\
ERA  & Eigensystem realization algorithm \\
ESC  & Extremum-seeking control \\
FFT  & Fast Fourier transform \\
GMM  & Gaussian mixture model \\
HAVOK  & Hankel alternative view of Koopman \\
ICA  & Independent component analysis \\
JL  & JohnsonLindensfrauss \\
KL  & KullbackLeib1er \\
KLT  & Karhunen-Loeve transform \\
LAD  & Least absolute deviations \\
LASSO  & Least absolute shrinkage and selection operator \\
LDA  & Linear discriminant analysis \\
LQE  & Linear quadratic estimator \\
LQG  & Linear quadratic Gaussian controller \\
LQR  & Linear quadratic regulator \\
LTI  & Linear time invariant system \\
MIMO  & Multiple input, multiple output \\
MLC  & Machine learning control \\
Most  & Common Acronyms \\
MPE  & Missing point estimation \\
mrDMD  & Multi-resolution dynamic mode decomposition \\
NARMAX  & Nonlinear autoregressive model with exogenous inputs \\
NLS  & Nonlinear Schrdinger equation \\
ODE  & Ordinary differential equation \\
OKID  & Observer Kalman filter identification \\
PBH  & PopovBelevitchHautus test \\
PCA  & Principal components analysis \\
PCP  & Principal component pursuit \\
PDE  & Partial differential equation \\
PDE-FIND  & Partial differential equation functional identification of nonlinear dynamics \\
PDF  & Probability distribution function \\
PID  & Proportional-integral-derivative control \\
PIV  & Particle image vetocimetry \\
POD  & Proper orthogonal decomposition \\
RIP  & Restricted isometry property \\
RKHS  & Reproducing kernel Hilbert space \\
RNN  & Recurrent neural network \\
ROM  & Reduced order model \\
RPCA  & Robust principal components analysis \\
rSVD  & Randomized SVD \\
SGD  & Stochastic gradient descent \\
SINDy  & Sparse identification of nonlinear dynamics \\
SISO  & Single input, single output \\
SRC  & Sparse representation for classification \\
SSA  & Singular spectrum analysis \\
STFT  & Short time Fourier transform \\
STLS  & Sequential thresholded least-squares \\
SVD  & Singular value decomposition \\
SVM  & Support vector machine \\
TICA  & Time-lagged independent component analysis \\
VAC  & Variational approach of conformation dynamics \\
\hline
\end{tabular}
\end{center}
\end{table}


\Gls{computer}
