\begin{enumerate}
\item At end of FY 20/21, a report describing the important regimes in the 
tokamak plasma edge (relevant to project \nep) and a review of the 
state-of-the-art in their theoretical modelling, including prioritisation of 
key physical processes for modelling and for further research. At the end of FY 
21/22, a report describing models produced as a result of the further research.
An indicative approach involves the production of equations for
individual models separately in magnetic field geometries of increasing complexity, 
enumerated above as~(i)-(iv), beginning in the slab~(i) in the limits of
\begin{enumerate}
\item drift-kinetic physics (SOL, small drift-ratio)
\item high collisionality (far SOL)
\item small gradients (hot plasma)
\end{enumerate}
The next step might be to explore efficient
representation of the sheath to provide approximate boundary
conditions for the theoretical  models.
\item At end of  FY 20/21, a report containing material to assist implementation of the priority 
models in software, including discrete conservation and other relations useful for code 
validation, and indicating cases suitable for comparison studies with other 
tokamak edge codes such as GKEYLL~\cite{Sh19Full}. This report should be a ``living document",
to be updated in the light of subsequent research.
\item \Papp s to demonstrate feasibility of model implementation at the Exascale. 
In particular, \papp s should enable estimation of efficiency and accuracy achievable by the 
proposed model in a realistic, detailed calculation.
An indicative approach involves the production of \papp s for 
individual models separately as indicated above, ie.\ in the limits of
\begin{enumerate}
\item drift-kinetic physics (SOL)
\item high collisionality (far SOL)
\item small gradients (hot plasma)
\end{enumerate}
and thereafter implementing sheath boundary conditions.
\item Subject to prior agreement, the bidder may derive model equations
or work in magnetic field geometries
presenting equal or greater relevant challenges to those described above,
for implementation in the corresponding \papp.
\end{enumerate}

