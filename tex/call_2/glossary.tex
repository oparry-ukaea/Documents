\begin{table}[h]
\textbf{\textsf{GLOSSARY:}}
\begin{center}
\begin{tabular}{|p{4.0cm}|p{12.0cm}|}
\hline
\textbf{\textsf{Term or Acronym}}
& \textbf{\textsf{Definition}} \\
\hline
 Nektar++ & Spectral/hp element library \url{https://www.nektar.info/}\\
PIC & Particle-in-Cell \\
SPH & Smoothed Particle Hydrodynamics \\
QMC & Quasi-Monte-Carlo \\
\hline
`trapped-passing' boundary & There are two important classes of particle
trajectories, depending on the particle energy and where in the magnetic field they begin.
The first `passing' set approximately follow the fieldlines, gyrating as they go, whereas
the second `trapped' set have trajectories that appear to `bounce'. \\ 
\hline
\end{tabular}
\end{center}
\end{table}
%\section*{References}
\bibliographystyle{unsrt}
\bibliography{../bib/new,../bib/waynes,../bib/misc,../bib/warv,../bib/neuts,../bib/reac,../bib/exc,../bib/active,../bib/dg1srt}
