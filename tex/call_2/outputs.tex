\begin{enumerate}

\item There are several potentially disruptive approaches to particle
modelling that need
evaluating before producing a detailed software design. These are 
\begin{itemize}
\item Timestep-robust particle tracking to ameliorate~(i)
\item Use of Quasi-Monte-Carlo techniques and related sampling techniques for (ii)
\item Particle or kinetic enslavement to help with~(iii)
\end{itemize}
Another potentially disruptive approach may be identified by the bidder.
The demands of particle storage and management in memory will determine how \emph{all} field data
is assigned to memory, and this aspect must also addressed.


\item At least three potentially disruptive approaches will be evaluated by means of 
literature and code surveys and consultation
with UKRI and industry experts. For this exercise the report
CD/EXCALIBUR-FMS/0013 provides initial pointers, but is unlikely
to be comprehensive.
Relatively small development tasks may initially be undertaken to 
test candidate methods for
accuracy, stability and HPC scalability potential.
At minimum, evaluation should produce an objective assessment as to
whether an approach is suitable for:
\begin{itemize}
\item Immediate implementation in \nep\ .
\item Recognition by the \nep\ design so that deferred implementation is facilitated.
\item Implementation only in a restricted class of machine architectures,
or inappropriate.
\end{itemize}

\item The bidder should indicate how, over the duration of the grant, they intend to engage
in community building and development, for example by assisting in the production
of one or more \papp s to demonstrate key features of preferred approaches.


\end{enumerate}

%\input{../commitment0}

