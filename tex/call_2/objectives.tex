Sheath formation leads to the need to model the sheath plasma using particles,
typically with the Particle-in-Cell~(PIC) approach, while being able to treat adjacent
plasma as a fluid.
In fact it may be easier at the Exascale to use a particle representation
for at least some ionised species throughout most of the edge plasma volume,
and possibly efficient because the distribution function in velocity may not be
smooth  at the `trapped-passing' boundary.
Indeed, the need to mix particle as well as fluid models
gives rise to an objective  to be able to move between representations without
introducing excessive error in time, as well as space.

Neutrals are generically less likely than charged species to thermalize.
They circulate into hotter and denser regions and collide with charged species.
Since the neutrals are generally far from thermal equilibrium, it is natural
to represent them using a particle model throughout the volume,
with an objective to calculate their interaction with plasma without
significant violation of mass, momentum and energy conservation on either side.

All the above effects and objectives require use of particle methods
associated until now with well-known deficiencies concerning (i) the very short
gyro-orbit timescale for plasma, (ii) a slow rate of convergence due to
the Monte-Carlo sampling which they represent, and (iii) the need to use
many more particles than grid points which makes implicit treatments
very expensive.
\emph{N.B. Discussion in this note implies usage of particles to model kinetic effects,
not schemes such as SPH designed to model advection of classical fluids.
Exploration of the concept of phase-fluid, whereby extra dimensions represent the velocity-space
dependence at a position in full detail, is not expected in response to this part of the call.}



