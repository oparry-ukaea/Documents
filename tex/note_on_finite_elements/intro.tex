Most scientists are trained at university in the finite difference~(fd)
method of approximating solutions to partial differential equations.
This permits focus on the properties of fd schemes and physics
in simple configurations such as cuboids or circular cylinders.
However, in their later career scientists almost inevitably come across
more complex geometrical figures, and 
%either through ignorance or arrogance
seek to adapt their fd knowledge to this situation,
typically re-inventing some (very) old and wobbly wheels in the process.

This note in \Sec{meshing} explains in simple terms why in general the finite element
approach should be preferred to idiosyncratic adaptations,
even though it implies the need to
learn about importing CAD generated geometry, meshing tools and implicit solvers.
One caveat is that the nomenclature for the various approaches listed
is not fully standardised, however it should hopefully be evident
from the description in each section, which idea is being explored.

The note goes on in \Sec{sem} to describe
other distinctive features of the finite element
approach, namely the global nature of the approximation and the 
need to invert large sparse matrices efficiently.
These are discussed with particular emphasis on
spectral elements.

%Note that this note does \emph{not} pursue in detail the further advantages
%of the finite element
%approach, that it defines an approximation for the solutions interior
%to arbitrarily shaped ``elements", as well as methods for computing them.
