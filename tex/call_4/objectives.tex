The \nep\ objectives that a ROM could help meet are
\begin{enumerate}
\item To assist in UQ by enabling faster solution of a problem at a wider range of
parameters than a full model would allow.
(Note that there is a separate call for UQ.)
\item To assist in validation by helping users interpret simulation results.
\item To reduce the rate at which data needs to be transferred (data compression)
either between processors or to disc.
\item To convert between models of different complexity, eg.\ as
given by number of spatial dimensions.
\item (Of course, the ROM might also be independently be useful say in real-time
control outside the main simulations.) 
\end{enumerate}
The problems that would benefit from ROM are mostly nonlinear and exhibit turbulence.

Achieving any of the above objectives would be useful, but especially
if the ROM is used in pursuit of numbers~1~or~3,
its errors must be preferably quantifiable or least be
bounded in some useful way, to ensure actionable simulation.
At the Exascale, the method employed to produce the ROM should itself seek
to minimise data transfer.

