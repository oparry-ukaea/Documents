\begin{itemize}
\item This task focuses upon the suitability of available numerical 
algorithms (or the development of new algorithms)
for Exascale targeted UQ. As stated in the Fusion Modelling System
Science Plan
\begin{itemize}
\item Quality control, verification, validation and uncertainty quantification
(VVUQ, eg.\ via intrusive or ensemble-based methods) will be embedded across all
areas of the project to ensure numerical predictions are ``actionable''. 
\item Identifying, quantifying and mitigating uncertainty and the
computational overhead of adequately modelling each will be a core theme within the project.
\item The overall aim, 
\ldots of the project \ldots is to build a hierarchy of models that are capable of
representing edge plasma behaviour to within a specific level of uncertainty.
\end{itemize}

\item The ideal numerical algorithms for forming the Exascale edge plasma codes of 
the future will have preferably at least the following properties:
\begin{itemize}
\item[{\bf P1}] Accurate solution of hyperbolic problems.
%\item[{\bf P2}] Ability to deliver efficient and accurate solutions of corresponding 
%elliptic problems.
\item[{\bf P3}] Accurate modelling of highly anisotropic dynamics. 
%\item[{\bf P4}] Accurate representation of first wall geometry (face normals to
%within~$0.1^{0}$), and correspondingly of complex magnetic field geometries.
\item[{\bf P5}] Accurate representation of velocity (phase) space.
%\item[{\bf P6}] Preservation of conservation properties of the underlying equations.
\item[{\bf P7}] Scalability to likely Exascale architectures:
\begin{itemize}
\item[a] interaction between models of different dimensionality,
\item[b] interaction between particle and fluid models,
\item[c] dynamic construction of surrogates.
\end{itemize}
\item[{\bf P8}] Performance portability to allow rapid deployment upon emerging hardware.
\end{itemize}

\item It is unlikely that any algorithm will be optimal in all categories,
and part of the exercise
will be to rank the importance of these properties.
\end{itemize}

\begin{enumerate}
\item Intrusive Uncertainty Quantification~(UQ) implies
typically PC or gPC, whereby probability distributions
of fields are expanded in terms of Hermite polynomials
or other orthogonal polynomials respectively.
\item Semi-intrusive Uncertainty Quantification~(UQ) implies either
\begin{itemize}
\item Probability distributions expanded in terms of
other basis functions than PC or gPC.
\item In a multi-scale or multi-level calculation, different
sampling rates or techniques at different levels, typically
fewer calculations at microscale.
\end{itemize}
\item Non-intrusive UQ typically implies the construction of ensembles
of realisations of the model.
\end{enumerate}

Suggested sources
\begin{enumerate}
\item R.C. Smith, SIAM 2013, ``Uncertainty quantification: theory, implementation, and applications"
\item Report of INI 2018 Programme on Uncertainty Quantification \\
\url{https://www.newton.ac.uk/files/reports/scientific/unq.pdf}
%\item C.G. Albert for indicative fusion case, UQ of plasma profiles
\end{enumerate}

