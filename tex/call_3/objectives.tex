The outputs of the software will be used in procurements of
first wall hardware that may involve tens if not hundreds of millions
of pounds. Two things are therefore essential, (1) that the numerical
results have quantified accuracy regarding both round-off and truncation
error, and (2) that the robustness of the simulation results with
respect to small changes in the modelling parameters be established.
The latter is particularly challenging not only because of the 
complexity of the physical models but also the likely presence of turbulence.

The former~(1) will largely be dealt with in other calls, but must be
borne in mind when treating~(2). Three approaches appear to be recognised
to UQ, namely non-intrusive, semi-intrusive
(which sadly appears to be admit at least two different definitions)
and intrusive. At the Exascale, even non-intrusive UQ may have challenges
as the ensemble of calculations may generate potentially unmanageable
amounts of data, one of the answers to which is Model Order Reduction,
the subject of a separate call.

Semi-intrusive and intrusive techniques for UQ are required
which are efficient in the case of both relatively small (less
than approx.~$10$) numbers of parameters
and also large numbers of parameters that give
rise to the so-called  `curse of dimensionality'.
Generally, different techniques will be required in
the two limits, and for example one approach, Inverse Regression, involve
first identification of a small number of important parameters
from the larger set. There could be benefits if the same basis
functions were used in both probability space and physical space
(including velocity space).

