\section{Advection and Diffusion Separately}\label{sec:work}
\subsection{Pure Advection}\label{sec:puread}
Assume first that there is no diffusion $\kappa=0$.
Consider first a matrix derived by central differncing
\begin{equation}\label{eq:mat1}
A=\begin{pmatrix}
0 & 1 & 0 & 0 & 0 & 0\\
-1 & 0 & 1 & 0 & 0 & 0\\
0 & -1 & 0 & 1 & 0 & 0 \\
0 & 0 & -1 & 0 & 1 & 0\\
0 & 0 & 0 & -1 & 0 & 1\\
0 & 0 & 0 & 0 & -1 & 0
\end{pmatrix}
\end{equation}
Using reduce-algebra (CSL reduce), the inverse of centred-difference~$A$ is easily calculated as
\begin{equation}\label{eq:mat2}
\begin{pmatrix}
0 & -1 & 0 & -1 & 0 & -1\\
1 & 0 & 0 & 0 & 0 & 0\\
0 & 0 & 0 & -1 & 0 & -1 \\
1 & 0 & 1 & 0 & 0 & 0\\
0 & 0 & 0 & 0 & 0 & -1\\
1 & 0 & 1 & 0 & 1 & 0
\end{pmatrix}
\end{equation}
Note the way that odd and even columns and rows decouple.
Now by the arguments regarding
the application of Dirichlet boundary conditions in \Sec{intro},
the inverse matrix~\Eq{mat2} corresponds to a problem \Eq{mat1}  with~$f$ specified
at both inflow and outflow boundaries, which is peculiar from a physical standpoint.
Solving \Eq{mat1} by elimination - or by inspection of \Eq{mat2} - shows that
values~$f_2$, $f_4$, \ldots, corresponding to even-numbered rows are determined
from $f_0$ (assumed folded into $b_2$).
Similarly, since the matrix is $6\times6$, it is the boundary value~$f_7$ (as part of $b_5$)
that determines $f_5$, $f_3$ and $f_1$ in a sequence of decreasing odd index.

This illustrates how arises the well-known difficulty with centred differences
applied to advection-diffusion that the odd and even numbered variables tend to 
decouple, resulting in chequerboard instabilities in 2-D. Note that if $A$ is of
size given by an odd number, it is singular, because both boundary constraints
fall on the even-numbered field samples $f_2$, $f_4$ etc.\ and and none apply
to the odd-numbered ones.

A common way to avoid such issues is to upwind, see eg.\ the books by Peyret
and Taylor~\cite{peyrettaylor} or Patankar~\cite{patankar}.
With the flow (`wind') blowing from
negative~$x$, apply upwinding only  at the outlet boundary $x=L$, this gives a matrix
\begin{equation}\label{eq:matcu1}
\begin{pmatrix}
0 & 1 & 0 & 0 & 0 & 0 & 0\\
-1 & 0 & 1 & 0 & 0 & 0 & 0\\
0 & -1 & 0 & 1 & 0 & 0 & 0\\
0 & 0 & -1 & 0 & 1 & 0 & 0\\
0 & 0 & 0 & -1 & 0 & 1 & 0\\
0 & 0 & 0 & 0 & -1 & 0 & 1\\
0 & 0 & 0 & 0 & 0 & -2 & 2
\end{pmatrix}
\end{equation}
The factors of two in the last row follow because the discrete derivative
is formed by the difference between adjacent mesh values. In this case, and
indeed whether the size of~$A$ is odd or even,  its inverse exists
and indeed provides a degree of coupling between the odd and even meshes,
being for the $7 \times 7$ case:
\begin{equation}\label{eq:matcui1}
\begin{pmatrix}
1 & -1 & 1 & -1 & 1 & -1 & \frac{1}{2}\\
1 & 0 & 0 & 0 & 0 & 0 & 0\\
1 & 0 & 1 & -1 & 1 & -1 & \frac{1}{2}\\
1 & 0 & 1 & 0 & 0 & 0 & 0\\
1 & 0 & 1 & 0 & 1 & -1 & \frac{1}{2}\\
1 & 0 & 1 & 0 & 1 & 0 & 0\\
1 & 0 & 1 & 0 & 1 & 0 & \frac{1}{2}\\
\end{pmatrix}
\end{equation}

Applying upwinding throughout the domain gives a matrix
\begin{equation}\label{eq:mat3}
A=\begin{pmatrix}
1 & 0 & 0 & 0 & 0 & 0\\
-1 & 1 & 0 & 0 & 0 & 0\\
0 & -1 & 1 & 0 & 0 & 0 \\
0 & 0 & -1 & 1 & 0 & 0\\
0 & 0 & 0 & -1 & 1 & 0\\
0 & 0 & 0 & 0 & -1 & 1
\end{pmatrix}
\end{equation}
with inverse
\begin{equation}\label{eq:mat4}
\begin{pmatrix}
1 & 0 & 0 & 0 & 0 & 0\\
1 & 1 & 0 & 0 & 0 & 0\\
1 & 1 & 1 & 0 & 0 & 0 \\
1 & 1 & 1 & 1 & 0 & 0\\
1 & 1 & 1 & 1 & 1 & 0\\
1 & 1 & 1 & 1 & 1 & 1
\end{pmatrix}
\end{equation}
Note that this has a much different structure, lower triangular, compared to the inverse in \Eq{mat2}.
It clearly reflects the flow physics of the problem, with downstream values only
depending on upstream ones. Similarly if the flow is directed from the opposite direction,
then the matrix~$A$ becomes the negative transpose of that in \Eq{mat3} and
correspondingly its inverse is upper triangular
\begin{equation}\label{eq:mat5}
\begin{pmatrix}
-1 & -1 & -1 & -1 & -1 & -1\\
0 & -1 & -1 & -1 & -1 & -1\\
0 & 0 & -1 & -1 & -1 & -1\\
0 & 0 & 0 & -1 & -1 & -1\\
0 & 0 & 0 & 0 & -1 & -1\\
0 & 0 & 0 & 0 & 0 & -1
\end{pmatrix}
\end{equation}

\subsection{Pure Diffusion}\label{sec:puredi}
The next case neglects flow and assumes nonzero constant~$\kappa=$, so that close to steady state
the matrix for a second-order centred discretisation of the purely diffusive problem is
\begin{equation}\label{eq:mat7}
D=\begin{pmatrix}
-2 & 1 & 0 & 0 & 0 & 0\\
1 & -2 & 1 & 0 & 0 & 0\\
0 & 1 & -2 & 1 & 0 & 0 \\
0 & 0 & 1 & -2 & 1 & 0\\
0 & 0 & 0 & 1 & -2 & 1\\
0 & 0 & 0 & 0 & 1 & -2
\end{pmatrix}
\end{equation}
with normalised inverse (apply factor~$1/7$) of
\begin{equation}\label{eq:mat6}
-D^{-1}=\begin{pmatrix}
6 & 5 & 4 & 3 & 2 & 1\\
5 & 10 & 8 & 6 & 4 & 2\\
4 & 8 & 12 & 9 & 6 & 3\\
3 & 6 & 9 & 12 & 8 & 4\\
2 & 4 & 6 & 8 & 10 & 5 \\
1 & 2 & 3 & 4 & 5 & 6
\end{pmatrix}
\end{equation}
Note that consistent with the physically expected  effects of diffusion, the inverse has largest entries on
the diagonal, and its entries become smaller with distance from the diagonal. \Eq{mat6} may
also be regarded as an analogue of the discrete Green's function.

\subsection{Advection with flow reversal}\label{sec:reverse}
A more \nep-relevant problem is driven by a source at the domain midpoint~$x=L/2$, so that the
flow reverses direction there. Unlike centred differences, the upwind scheme matrix~$A$ is
sensitive to flow direction. Moreover for the scheme to be conservative, a staggered grid
is typically used, so that the flow has sample values $u_{i\pm 1/2}$ at points
intermediate between those at which $f_i$ is defined. There are four
cases~\cite[\S\,5.2-2]{patankar} depending
on the signs of $u_{i\pm 1/2}$, the two already implicitly covered when the signs are
the same, and two when they are opposite representing respectively converging and diverging
flow at the sample point of~$f_i$, so that the coefficients of $f_{i-1}$, $f_i$ and~$f_{i+1}$
may be enumerated as 3-vectors as follows
\begin{enumerate}
\item Forwards flow $u_{i-1/2},\;\; u_{i+1/2} > 0$,
\begin{equation}\begin{pmatrix}
-u_{i-1/2} & u_{i+1/2} & 0
\end{pmatrix} \end{equation}
\item Converging flow $u_{i-1/2}>0>u_{i+1/2}$,
\begin{equation}\begin{pmatrix}
-u_{i-1/2} & 0 & u_{i+1/2} 
\end{pmatrix} \end{equation}
\item Diverging flow $u_{i+1/2}>0>u_{i-1/2}$,
\begin{equation}\begin{pmatrix}
0 & u_{i+1/2}-u_{i-1/2} & 0
\end{pmatrix} \end{equation}
\item Backwards flow $u_{i-1/2},\;\; u_{i+1/2} < 0$,
\begin{equation}\begin{pmatrix}
0 & -u_{i-1/2} & u_{i+1/2}
\end{pmatrix} \end{equation}
\end{enumerate}
Consider an even case where $A$ is of size~$6\times6$. This corresponds
to an odd number~$5$ of flow samples~$u_{i\pm 1/2}$, let them be described by the
vector of samples
\begin{equation} \label{eq:matdi0}
\begin{pmatrix}
u_{1/2} & u_{3/2} & \ldots 
\end{pmatrix}
=
\begin{pmatrix}
-1 & -u & 0 & v & 1
\end{pmatrix}
\end{equation}
where $u>0$ and $v>0$ so that flow diverges about the point~$5/2$, then the above upwind rules give
\begin{equation}\label{eq:matdi1}
A=\begin{pmatrix}
1 & -1 & 0 & 0 & 0 & 0\\
0 & 1 & -u & 0 & 0 & 0\\
0 & 0 & -u & 0 & 0 & 0\\
0 & 0 & 0 & v & 0 & 0\\
0 & 0 & 0 & -v & 1 & 0\\
0 & 0 & 0 & 0 & -1 & 1
\end{pmatrix}
\end{equation}
with determinant $det(A)=uv$ and inverse
\begin{equation}\label{eq:matdi2}
\begin{pmatrix}
1 & 1 & 1 & 0 & 0 & 0\\
0 & 1 & 1 & 0 & 0 & 0\\
0 & 0 & \frac{1}{u} & 0 & 0 & 0\\
0 & 0 & 0 & \frac{1}{v} & 0 & 0\\
0 & 0 & 0 & 1 & 1 & 0\\
0 & 0 & 0 & 1 & 1 & 1
\end{pmatrix}
\end{equation}
However for $A$ of size $5\times5$, with a diverging flow about the point~$2$,
\begin{equation} \label{eq:matodi0}
\begin{pmatrix}
u_{1/2} & u_{3/2} & \ldots 
\end{pmatrix}
=
\begin{pmatrix}
-1 & -u &  v & 1
\end{pmatrix}
\end{equation}
the above upwind rules give
\begin{equation}\label{eq:matdi3}
A=\begin{pmatrix}
1 & -1 & 0 & 0 & 0\\
0 & 1 & -u & 0 & 0\\
0 & 0 & u+v & 0 & 0\\
0 & 0 & -v & 1 & 0\\
0 & 0 & 0 & -1 & 1
\end{pmatrix}
\end{equation}
with determinant $det(A)=u+v$ and inverse
\begin{equation}\label{eq:matd4}
\begin{pmatrix}
1 & 1 & \frac{u}{(u+v)} & 0 & 0\\
0 & 1 & \frac{u}{(u+v)} & 0 & 0\\
0 & 0 & \frac{1}{(u+v)} & 0 & 0\\
0 & 0 & \frac{v}{(u+v)} & 1 & 0\\
0 & 0 & \frac{v}{(u+v)} & 1 & 1
\end{pmatrix}
\end{equation}
It will be seen that if $u$~and/or~$v$ are small in some sense, then the conditioning of~$A$
with odd dimension may be much better than with even. Fundamentally of course the magnitude of
determinant depends on the location of the stagnation points relative to the offset or~`$1/2$'-grid.

\section{Advection-diffusion}\label{sec:addi}
In many cases of interest, diffusion is small relative to advection so that the Peclet
number~$Pe=||{\bf u}||  L/\kappa$ (where the domain of interest is $0<x<L$) is large.
The behaviour of solutions to the discrete problem is governed by the dimensionless
group known as the grid or mesh Peclet number $Pe_h=||{\bf u}|| h/\kappa$ where $h$~is the
mesh spacing implicitly assumed to be uniform, ie.\  $h=L/(N-1)$ where $N\times N$~is
the size of~$A$ or~$D$. Suppose $Pe_h=1/\epsilon$, then the matrix $A-\epsilon D$ 
requires inversion, for different choices of schemes.

Provided $A$ is well-behaved, then the small perturbation due to conductivity should make
little difference. Hence, interest attaches to the centred difference case, and to the
divergent upwind case where $u$~and~$v$ might be considered small of order~$\epsilon$
near the stagnation point.

\subsection{Centred Difference Scheme}\label{sec:centred}
As might be expected, the addition of a small diffusive term makes little difference~$\mathcal{O}(\epsilon)$
to  the determinant of the even-sized matrix, which changes to $1+21 \epsilon^2+ \mathcal{O}(\epsilon^4)$,
with entries accurate to $\mathcal{O}(\epsilon^3)$ of:
\begin{equation}\begin{pmatrix}
6 \epsilon & -1+\epsilon-10 \epsilon^2 & 4 \epsilon-8 \epsilon^2 & -1+3 \epsilon-6 \epsilon^2 & 2 \epsilon-8 \epsilon^2 & -1+5 \epsilon-10 \epsilon^2  \\
1+\epsilon+10 \epsilon^2 & 2 \epsilon & -8 \epsilon^2 & 2 \epsilon-4 \epsilon^2 & -4 \epsilon^2 & 2 \epsilon-8 \epsilon^2  \\
4 \epsilon+8 \epsilon^2 & 8 \epsilon^2 & 4 \epsilon & -1+\epsilon-6 \epsilon^2 & 2 \epsilon-4 \epsilon^2 & -1+3 \epsilon-6 \epsilon^2  \\
1+3 \epsilon+6 \epsilon^2 & 2 \epsilon+4 \epsilon^2 & 1+\epsilon+6 \epsilon^2 & 4 \epsilon & -8 \epsilon^2 & 4 \epsilon-8 \epsilon^2  \\
2 \epsilon+8 \epsilon^2 & 4 \epsilon^2 & 2 \epsilon+4 \epsilon^2 & 8 \epsilon^2 & 2 \epsilon & -1+\epsilon-10 \epsilon^2  \\
1+5 \epsilon+10 \epsilon^2 & 2 \epsilon+8 \epsilon^2 & 1+3 \epsilon+6 \epsilon^2 & 4 \epsilon+8 \epsilon^2 & 1+\epsilon+10 \epsilon^2 & 6 \epsilon 
\end{pmatrix} \end{equation}
One interesting point is that although the odd source
terms~$b_{2j+1}$ now couple to even indexed~$f_i$ and vice versa, the corresponding entries in the matrix
may frequently be as small as~$\mathcal{O}(\epsilon^2)$.

The diffusion term makes the odd-sized matrix 
invertible, albeit with small $det(A)=6\epsilon+20\epsilon^3+\mathcal{O}(\epsilon^5)$,
with entries accurate to $\mathcal{O}(\epsilon^3)$ of:
\begin{equation}\begin{pmatrix}
1+10 \epsilon^2 & -4 \epsilon+4 \epsilon^2 & 1-2 \epsilon+4 \epsilon^2 & -2 \epsilon+6 \epsilon^2 & 1-4 \epsilon+6 \epsilon^2  \\
4 \epsilon+4 \epsilon^2 & 8 \epsilon^2 & -2 \epsilon+2 \epsilon^2 & 4 \epsilon^2 & -2 \epsilon+6 \epsilon^2  \\
1+2 \epsilon+4 \epsilon^2 & 2 \epsilon+2 \epsilon^2 & 1+6 \epsilon^2 & -2 \epsilon+2 \epsilon^2 & 1-2 \epsilon+4 \epsilon^2  \\
2 \epsilon+6 \epsilon^2 & 4 \epsilon^2 & 2 \epsilon+2 \epsilon^2 & 8 \epsilon^2 & -4 \epsilon+4 \epsilon^2  \\
1+4 \epsilon+6 \epsilon^2 & 2 \epsilon+6 \epsilon^2 & 1+2 \epsilon+4 \epsilon^2 & 4 \epsilon+4 \epsilon^2 & 1+10 \epsilon^2 
\end{pmatrix} \end{equation}
In summary, despite the presence of diffusion, 
the odd-even coupling remains weak in both odd and even cases, underlining the fact that $A$~may
be well-conditioned without giving physically well-behaved results.

\subsection{Patankar Scheme}\label{sec:patankar}
Strictly speaking, there is a family of Patankar schemes~\cite[\S\,5.2]{patankar}, each member sharing the basic idea that each
becomes the above upwind schemes at large~$Pe_h$ and centred differences as 
$Pe_h\rightarrow 0$.  Nonetheless, the scheme~\cite[\S\,5.2-5]{patankar} that simply switches between upwind and
centered at $Pe_h=2$ is found to have comparable performance to more sophisticated 
variants~\cite[\S\,5.2-6]{patankar}. To simplify notation, suppose that flows are normalised by the outflow
speed~$U_0=||{\bf u}||$, and length by~$h$, so that modifications from upwind
are needed whenever flow  speed drops below~$2\epsilon$. 
There are three separate cases depending as to whether (1) both $|u_{i\pm 1/2}|<2\epsilon$,
(2) $|u_{i-1/2}|< 2\epsilon$ only and (3) $|u_{i+1/2}|< 2\epsilon$ only, and then separate subcases
depending on the signs of $u_{i\pm 1/2}$, except in the first case.

\begin{enumerate}
\item When both $|u_{i\pm 1/2}|<2\epsilon$, centred differences apply. 
The coefficients of $f_{i-1}$, $f_i$ and~$f_{i+1}$ as 3-vectors are
\begin{enumerate}
\item Any flow $u_{i-1/2},\;\; u_{i+1/2}$,
\begin{equation}\begin{pmatrix}
-\frac{1}{2}u_{i-1/2}-\epsilon & -\frac{1}{2}u_{i-1/2} + \frac{1}{2}u_{i+1/2}-2\epsilon & \frac{1}{2}u_{i+1/2}-\epsilon
\end{pmatrix} \end{equation}
\end{enumerate}
%the two already implicitly covered when the signs are
%the same, and two when they are opposite representing respectively converging and diverging
%flow at the sample point of~$f_i$, so that
\item When $|u_{i-1/2}|< 2\epsilon$ and $|u_{i+1/2}|> 2\epsilon$,
the coefficients of $f_{i-1}$, $f_i$ and~$f_{i+1}$
may be enumerated as 3-vectors as follows
\begin{enumerate}
\item Forwards flow $u_{i-1/2},\;\; u_{i+1/2} > 0$ and diverging flow $u_{i+1/2}>0>u_{i-1/2}$,
\begin{equation}\begin{pmatrix}
-\frac{1}{2}u_{i-1/2} -\epsilon & -\frac{1}{2}u_{i-1/2} -u_{i+1/2}+\epsilon & 0
\end{pmatrix} \end{equation}
\item Converging flow $u_{i-1/2}>0>u_{i+1/2}$ and backwards flow $u_{i-1/2},\;\; u_{i+1/2} < 0$,
\begin{equation}\begin{pmatrix}
-\frac{1}{2}u_{i-1/2} -\epsilon & -\frac{1}{2}u_{i-1/2} +\epsilon &  -u_{i+1/2}
\end{pmatrix} \end{equation}
\end{enumerate}
\item When $|u_{i-1/2}|> 2\epsilon$ and $|u_{i+1/2}|< 2\epsilon$,
the coefficients of $f_{i-1}$, $f_i$ and~$f_{i+1}$
may be enumerated as 3-vectors as follows
\begin{enumerate}
\item Forwards flow $u_{i-1/2},\;\; u_{i+1/2} > 0$ and diverging flow $u_{i+1/2}>0>u_{i-1/2}$,
\begin{equation}\begin{pmatrix}
-u_{i-1/2} & \frac{1}{2}u_{i+1/2}+\epsilon & \frac{1}{2}u_{i+1/2}-\epsilon
\end{pmatrix} \end{equation}
\item Converging flow $u_{i-1/2}>0>u_{i+1/2}$ and backwards flow $u_{i-1/2},\;\; u_{i+1/2} < 0$,
\begin{equation}\begin{pmatrix}
0 & -u_{i-1/2}+\frac{1}{2}u_{i+1/2}+\epsilon & \frac{1}{2}u_{i+1/2}-\epsilon
\end{pmatrix} \end{equation}
\end{enumerate}
\end{enumerate}

Consider an even case where $A$ is of size~$6\times6$. This corresponds
to an odd number~$5$ of flow samples~$u_{i\pm 1/2}$, let them be described by the
vector of samples
\begin{equation} \label{eq:patdi0}
\begin{pmatrix}
u_{1/2} & u_{3/2} & \ldots 
\end{pmatrix}
=
\begin{pmatrix}
-1 & -u & \delta  & v & 1
\end{pmatrix}
\end{equation}
where $u>0$ and $v>0$ so that flow diverges about a  point near~$5/2$ such
that~$|\delta| < \epsilon$, then 
suppose first that both $u,\;v>2\epsilon$, then the above Patankar scheme gives
\begin{equation}\label{eq:patdi1}
A=\begin{pmatrix}
1 & -1 & 0 & 0 & 0 & 0\\
0 & 1 & -u & 0 & 0 & 0\\
0 & 0 & u+\frac{\delta}{2}+\epsilon & \frac{\delta}{2}-\epsilon & 0 & 0\\
0 & 0 & -\frac{\delta}{2}-\epsilon & v-\frac{\delta}{2}+\epsilon & 0 & 0\\
0 & 0 & 0 & -v & 1 & 0\\
0 & 0 & 0 & 0 & -1 & 1
\end{pmatrix}
\end{equation}
with $det(A)=uv+(\epsilon-\frac{1}{2}\delta)(u+v)$.
Now suppose that  both $|u|,\;|v|<2\epsilon$, then
\begin{equation}\label{eq:patdi1eps}
A=\begin{pmatrix}
1 & -1 & 0 & 0 & 0 & 0\\
0 & 1-\frac{u}{2}+\epsilon & -\frac{u}{2}-\epsilon & 0 & 0 & 0\\
0 & \frac{u}{2}-\epsilon & \frac{u+\delta}{2}+2\epsilon & \frac{\delta}{2}-\epsilon & 0 & 0\\
0 & 0 & -\frac{\delta}{2}-\epsilon & \frac{-\delta+v}{2}+2\epsilon & \frac{v}{2}-\epsilon & 0\\
0 & 0 & 0 & -\frac{v}{2}-\epsilon & 1-\frac{v}{2}+\epsilon & 0\\
0 & 0 & 0 & 0 & -1 & 1
\end{pmatrix}
\end{equation}
with $det(A)= \frac{1}{4} u v + 3 \epsilon^2 + \frac{1}{4} \delta (v-u) +
\frac{1}{2} \epsilon (2  u +  \delta +  2 v -u v - u \delta ) + 2 \epsilon^3 $

Turning to the case where $A$ is of size~$5\times5$. Let the even number of
$u_{i\pm 1/2}$ again be described by the vector 
\begin{equation} \label{eq:patdei0}
\begin{pmatrix}
u_{1/2} & u_{3/2} & \ldots 
\end{pmatrix}
=
\begin{pmatrix}
-1 & -u & v & 1
\end{pmatrix}
\end{equation}
then supposing both $|u|,\;|v|<2\epsilon$,
\begin{equation}\label{eq:patdei3}
A=\begin{pmatrix}
1 & -1 & 0 & 0 & 0\\
0 & 1-\frac{u}{2}+\epsilon & -\frac{u}{2}-\epsilon & 0 & 0\\
0 & \frac{u}{2}-\epsilon & \frac{u+v}{2} +2\epsilon & \frac{v}{2}-\epsilon & 0\\
0 & 0 & -\frac{v}{2}-\epsilon & 1-\frac{v}{2}+\epsilon & 0\\
0 & 0 & 0 & -1 & 1
\end{pmatrix}
\end{equation}
The inverse, normalised by $4 det(A)$ is
\tiny
\begin{equation}\label{eq:matidei3}
\begin{pmatrix}
2 (4 \epsilon^2 + 4 \epsilon - w) & 4 \epsilon^2 + 2 \epsilon u - 2 \epsilon v + 8 \epsilon + u + v - w & 4 \epsilon^2 + 2 \epsilon u - 2 \epsilon v + 4 \epsilon + u - v - w & 4 \epsilon^2 + 2 \epsilon u - 2 \epsilon v - u - v - w & 0\\
0 & 4 \epsilon^2 + 2 \epsilon u - 2 \epsilon v + 8 \epsilon + u + v - w & 4 \epsilon^2 + 2 \epsilon u - 2 \epsilon v + 4 \epsilon + u - v - w & 4 \epsilon^2 + 2 \epsilon u - 2 \epsilon v - u - v - w & 0\\
0 & 4 \epsilon^2 - 2 \epsilon u - 2 \epsilon v + 4 \epsilon - u + v + w & 4 \epsilon^2 - 2 \epsilon u - 2 \epsilon v + 8 \epsilon - u - v + w + 4 & 4 \epsilon^2 - 2 \epsilon u - 2 \epsilon v + 4 \epsilon + u - v + w & 0\\
0 & 4 \epsilon^2 - 2 \epsilon u + 2 \epsilon v - u - v - w & 4 \epsilon^2 - 2 \epsilon u + 2 \epsilon v + 4 \epsilon - u + v - w & 4 \epsilon^2 - 2 \epsilon u + 2 \epsilon v + 8 \epsilon + u + v - w & 0\\
0 & 4 \epsilon^2 - 2 \epsilon u + 2 \epsilon v - u - v - w & 4 \epsilon^2 - 2 \epsilon u + 2 \epsilon v + 4 \epsilon - u + v - w & 4 \epsilon^2 - 2 \epsilon u + 2 \epsilon v + 8 \epsilon + u + v - w & 2 (4 \epsilon^2 + 4 \epsilon - w)
\end{pmatrix}
\end{equation}
%\begin{pmatrix}
%2 (4 \epsilon^2 + 4 \epsilon - u v + u + v) & 4 \epsilon^2 + 2 \epsilon u - 2 \epsilon v + 8 \epsilon - u v + 2 u + 2 v & 4 \epsilon^2 + 2 \epsilon u - 2 \epsilon v + 4 \epsilon - u v + 2 u & 4 \epsilon^2 + 2 \epsilon u - 2 \epsilon v - u v & 0\\
%0 & 4 \epsilon^2 + 2 \epsilon u - 2 \epsilon v + 8 \epsilon - u v + 2 u + 2 v & 4 \epsilon^2 + 2 \epsilon u - 2 \epsilon v + 4 \epsilon - u v + 2 u & 4 \epsilon^2 + 2 \epsilon u - 2 \epsilon v - u v & 0\\
%0 & 4 \epsilon^2 - 2 \epsilon u - 2 \epsilon v + 4 \epsilon + u v - 2 u & 4 \epsilon^2 - 2 \epsilon u - 2 \epsilon v + 8 \epsilon + u v - 2 u - 2 v + 4 & 4 \epsilon^2 - 2 \epsilon u - 2 \epsilon v + 4 \epsilon + u v - 2 v & 0\\
%0 & 4 \epsilon^2 - 2 \epsilon u + 2 \epsilon v - u v & 4 \epsilon^2 - 2 \epsilon u + 2 \epsilon v + 4 \epsilon - u v + 2 v & 4 \epsilon^2 - 2 \epsilon u + 2 \epsilon v + 8 \epsilon - u v + 2 u + 2 v & 0\\
%0 & 4 \epsilon^2 - 2 \epsilon u + 2 \epsilon v - u v & 4 \epsilon^2 - 2 \epsilon u + 2 \epsilon v + 4 \epsilon - u v + 2 v & 4 \epsilon^2 - 2 \epsilon u + 2 \epsilon v + 8 \epsilon - u v + 2 u + 2 v & 2 (4 \epsilon^2 + 4 \epsilon - u v + u + v)
%\end{pmatrix}
%\end{equation}
\normalsize
where $w=uv-u-v$ and the determinant has increased to $det(A)=2\epsilon - \frac{1}{2}w  +2\epsilon^2=
\frac{1}{2}(u+v)  + 2\epsilon - \frac{1}{2}uv  +2\epsilon^2$.

Note the fact there are zeroes  in the first and last columns of the matrix~\Eq{matidei3}.
Experiments with larger matrices indicates that the inverse has indeed
the structure of  a `flow' into from the matrix diagonal as in eg.\ \Eq{matdi2}
combined with a diffusive spread over an extra entry above or below the diagonal.
The typical structure is illustrated by the pattern of non-zero entries~$X$ in the
$8\times8$ inverse as follows:
\begin{equation}\label{eq:matX}
\begin{pmatrix}
X & X & X & X & X & X & 0 & 0\\
0 & X & X & X & X & X & 0 & 0\\
0 & 0 & X & X & X & X & 0 & 0\\
0 & 0 & X & X & X & X & 0 & 0\\
0 & 0 & X & X & X & X & 0 & 0\\
0 & 0 & X & X & X & X & 0 & 0\\
0 & 0 & X & X & X & X & X & 0\\
0 & 0 & X & X & X & X & X & X
\end{pmatrix}
\end{equation}
