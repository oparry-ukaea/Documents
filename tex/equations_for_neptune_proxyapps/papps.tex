%The description of the \papp s in the Science Plan is inevitably
%terse and that in the document ``Equations for \exc/\nep \ \papp s"~\cite{pappeqs}
%is inevitably focussed on the mathematics.
%This appendix elucidates
%thinking behind their choice and the priorities for their implementation.

The overall thinking behind the \papp s is to explore potential `roadblocks' to the Exascale as
early and in as simple a context as is possible, beginning with algorithmic
roadblocks. \nep \ is directed towards producing `actionable' code as the basis
for large procurements, whereas
more physics-focussed software projects conducted by the
worldwide nuclear fusion have already advanced to greater complexity,
minimising the risk that unexpected problems will appear in the full model.
%is minimised by competing fusion software projects that will have `rushed in'
%to greater complexity, cf.\ the business adage that ``it is better to be second first".)

The numbering of \papp s below corresponds to the Science Plan.
\begin{itemize}
\item[2-1] 2-D model of anisotropic heat transport. It is important to determine early the
degree of anisotropy that high-order elements can treat without special coding. If this 
is unsatisfactorily small, then there are implications for geometry input as well as
algorithmic developments that are best addressed as early as possible.
\item[2-2] 2-D elliptic solver in complex geometry. One of the indicated elliptic solvers is
Grad-Shafranov to produce high order (`spectrally') accurate magnetic fields for
use in many other \papp s. Since Sovinec~\cite{Ho14Solv} has already produced a spectral element
Grad-Shafranov code, the corresponding \nep\ development should mainly serve to
identify practical issues concerning implementing high order fe models. The second
solver additionally presents a chance to explore comparatively novel meshing techniques developed under
Activity~A2.1, and later the preconditioning techniques of A2.7.
\item[2-3] 1-D fluid solver with simplified physics but with UQ and realistic boundary conditions.
This will determine the capability of spectral/hp element to handle sonic outflow
boundary conditions needed to represent sheaths, together with large source terms, as well
as identifying practical issues concerning intrusive UQ. This software is already
potentially useful in its own right for example in modelling MAST-U divertor,
and other workers might be drawn in to add additional physical effects to this end.
\item[2-4] Spatially 1-D plasma model incorporating velocity space effects. From the
numerical analytic point-of-view, this is a key demonstration of spectral/hp element capability to
handle particle interactions. % within the \emph{volume}.
However, again this could be a basis for divertor modelling,
to explore sheath effects depending on fieldline incidence on surface, and with minor
modification the spread of particle energy around tile edges and corners as performed
by Gunn~et~al~\cite{Gu19Stud,De19Phys} for ITER application.
\item[2-5] Spatially 1-D multispecies plasma model. Multispecies
throws up a surprising
number of issues concerning data definitions (eg.\ changes to the Coulomb
logarithm), structures to deal with different number of species, and perhaps most
significantly, complicated inter-species interaction terms both within
and at the domain boundaries. This is also an opportunity to mix fluid and
\emph{kinetic} representations of \emph{different} species  within the \emph{volume}.
\item[2-6] Spatially 2-D plasma model incorporating velocity space effects. 
With the 1-D multispecies fluid work's having made the generalisation to 2-D
straightforward, the challenge here is to start writing a complex \papp \
in production mode, incorporating the research put into design, documentation,
code generation and benchmarking. There is an opportunity to study species
with both fluid and \emph{kinetic} representations depending on location relative
to the wall. Again this is potentially a useful tool in its
own right, capable of revealing deficiencies in previous 2-D modelling work.
\item[2-7] Interaction between models of different dimensionality.
This should verify that the design has the right data structures to handle additional further
complexity beyond intrusive and ensemble-based UQ and model order reduction. The hopefully 
burgeoning \nep \ community
could develop this into a design tool with a capability both to explore a large
area of tokamak edge parameter space quickly in 0-D or 1-D and also to focus on
relatively small but critical 2-D features, such as tile edges.
\item[2-8] Spatially 3-D plasma \emph{kinetic} models.
These will represent the full fluid model
produced by the 5-year \nep \ project, incorporating features of 2-D fluid and
\emph{kinetic} work in a 3-D code.
\end{itemize}
\begin{itemize}
\item[3-1] 2-D particle-based model of neutral gas \& impurities with critical physics.
This will be a $2d3v$~code (ie.\ spatially 2-D distribution of particles
with $3$~velocity components)
designed from the outset to interact with a high-order finite-element fluid model of plasma.
It gives an opportunity to check out ideas on optimal usage of particles.
\item[3-2] 2-D moment-based model of neutral gas \& impurities.
Constructing a 2-D fluid code of neutral gas from the Nektar++ software should be
a valuable educational exercise, whilst providing scope for cross-validating the 2-D particle model.
\item[3-3] Interaction with 2-D plasma model when available.
Building on the 1-D multispecies fluid work, the challenge here is to incorporate
in the fluid code of \Papp \ PA2-6, particle effects from PA3-1, which will in
the higher dimensional space be more subject to lack of numerical
resolution or `noise'. Should PA3-3 be accelerated, it could usefully 
treat both plasma and neutrals via particle models.
\item[3-4] 3-D model of neutral gas \& impurities.
This is now at full dimensional complexity, incorporating selected ideas 
on optimal usage of particles.
\item[3-5] Interaction with 3-D plasma model.
This will represent the full model produced by the 5-year \nep \ project,
a coupling of fluid and \emph{kinetic} software developed under the FM-WP2 work-package as PA2-8,
incorporating features of \Papp s 3-1 to 3-4, and allowing for additional input from PA3-6.
\item[3-6] Staged introduction of additional neutral gas/impurity physics.
It is expected that the \nep \ community will join in to supplement
the software with a wide-ranging capability to treat a wide range of additional 
nuclear, atomic and molecular effects.
\end{itemize}


%Note: Work extending beyond Y3 is deliberately vague on the subject of gyrokinetics, as no
%widely accepted model for the tokamak edge appears to be available as of early~Y2,
%and even should one appear, it might not be suitable for use in~Y3 in~\nep.

