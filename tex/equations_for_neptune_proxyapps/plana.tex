In the Science Plan~\cite{sciplan} the description of work 
extending beyond Y3 (early 2022) is deliberately vague on the subject of ``gyrokinetics'', as no
widely accepted model for the tokamak edge appears to be available as of early~Y2,
and even should one appear, it might not be suitable for use in~Y3 in~\nep.

As listed, the \papp s correspond to Plan~A, which assumes that no suitable gyroaveraged
model will emerge in time, hence \emph{kinetic} implies Particle-in-Cell~(PIC).
PIC  approaches, where charge conservation
is vital to control errors, can anyway usefully be pursued
for modelling low collisionality plasma species, and should simplify nicely to 
treat neutral species with long mean-free-paths in tokamak edge problems
where mass conservation has been discovered to be
critical. Moreover in the context of classical fluid dynamics, 
the transition from fluid to particles, or short to long mean-free-path,
has been well-studied because of the
application to the space vehicle re-entry problem and related hypersonic situations.
Thus the hybrid fluid/PIC approach
might be regarded as a relatively low-risk route to achieving a robust
numerical algorithm.

Evidently full-orbit PIC has the potential to be extremely inefficient relative to 
gyroaveraged kinetic theory because of the need to follow gyro-orbits in detail. Hence
if this can be avoided, either through gyroaveraging or clever numerics or indeed
a combination of both,
then Plan~B will see \emph{kinetic} imply gyroaveraged kinetic theory for modelling plasma species
in the \papp s.

Regarding implementation at the Exascale there is also a
conservative Plan~A approach which sees the
use of relatively simple data structures such as scalar and vector arrays to transfer
data, and consequent use of existing code-coupling technologies.
Plan~B is an aggressive approach to implementation which sees
custom data structures allowing for all physical data (particle arrays
and fluid field vectors) colocated near a point to be
held close in memory, permitting very tight custom code-coupling. As with the
\emph{kinetic} options, this Plan~B promises significantly faster solutions than
the corresponding Plan~A, but its adoption depends on the outcome of research work
to de-risk.

