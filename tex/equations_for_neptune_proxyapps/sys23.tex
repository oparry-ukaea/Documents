\subsection{Plasma Equations}\label{sec:sys23plas}
It is assumed that the single spatial dimension of the problem
corresponds to the arc length distance along a fieldline.
Starting from the two-fluid model of Braginskii~~\cite{Br65Tranwarv},
a set of equations resembling those of classical (compressible) hydrodynamics
may be derived by summing Braginskii's equations for number density,
momentum and energy.
Using the notation of ref~\cite{Ha13Benc}, introducing $T_d=T_i+T_e$,
neglecting the stress tensor terms (implicitly setting $\delta p_i=0$),
and assuming $B$ is independent of time, the resulting system is
\begin{eqnarray}\label{eq:sysn1}
\frac{\partial}{\partial t} (N/B)+ 
\frac{\partial}{\partial s_{\|}} (N U/B)&=&\frac{S^n}{B} \\
\frac{\partial}{\partial t} (N U/B)+ 
\frac{\partial}{\partial s_{\|}} (N U^2/B)&=&
-\frac{1}{m_i B}\frac{\partial}{\partial s_{\|}} (p_i + p_e) +\frac{S^u}{m_i B} \label{eq:sysu1}\\
\frac{\partial}{\partial t}\left( \frac{3}{2}(N K_M T_d/B)+
\frac{1}{2} (N U^2/B) \right) &+& \nonumber \\
\frac{\partial}{\partial s_{\|}} \left( \frac{5}{2}(N U K_M T_d/B) +
\frac{1}{2} (N U^3/B) \right) &=&
-\frac{1}{m_i B}\frac{\partial}{\partial s_{\|}} (q_{i\|} + q_{e\|}) +
\frac{(S_i^\mathcal{E}+S_e^\mathcal{E})}{m_i B} \label{eq:syst1}
\end{eqnarray}
%$U_d =L_s/t_0$ is a speed measuring the importance of the transient term,
where $s_{\|}$ is distance along the fieldline,
%temporarily the conduction term in $(q_{i\|} + q_{e\|})$ has been retained,
and some variables from~\cite{Ha13Benc} have been promoted to capitals to 
indicate that they retain their physical dimensions.
For the case of a fieldline connecting two walls at~$s_{\|}=\pm L$,
suppose dimensionless~$s$, $0\leq s\leq 1)$ parameterises distance
\begin{equation}\label{eq:spar}
s_{\|}=L(2s-1)
\end{equation}
It is therefore helpful to make fieldline length dimensionless in terms of the distance
(assumed constant)
\begin{equation}\label{eq:hs}
L_s= \frac{d s_{\|}}{d  s}
\end{equation}
and so $L_s=2L$.
The constant $K_M$ is such that
\begin{equation}
K_M= \frac{k_B}{m_i}\;\;\mbox{or}\;\;K_M=\frac{|e|}{m_i}
\end{equation}
where $k_B$ is Boltzmann's constant and $|e|$ is the unit of charge,
depending whether T is measured in Kelvin or~$eV$. 
Note that in adding Eqs.(3) and~(4) of~\cite{Ha13Benc}, equipartition and collision terms cancel 
to give \Eq{syst1}. The perfect gas equation of state will be assumed, so that
\begin{equation}\label{eq:sysp}
\frac{p_i+p_e}{m_i}=N K_M T
\end{equation}

The thermal conduction fluxes are
\begin{equation}\label{eq:qei}
\frac{q_{\alpha\|}}{m_i}= -\kappa^\alpha_{\|} \frac{\partial K_M T_\alpha}{\partial s_{\|}},\;\;\;\alpha=e,\;i
\end{equation}
where the $\kappa^\alpha_{\|}$ take Braginskii values, see \Sec{bragin}, thus for usual situation
in which the electron conduction dominates
\begin{equation}\label{eq:qeisum}
\frac{q_{i\|} + q_{e\|}}{m_i}= \kappa^e_{\|} \frac{\partial K_M T_e}{\partial s_{\|}}
\end{equation}
and  $T_e$ has be expressed in terms of $T_d$, eg.\ $T_e=T_d/2$. To given an easier test
problem, this term is accounted for by augmenting the advective energy flux, $5/2\rightarrow g$.

The boundary conditions are that $|U|=|M_s| C_S$ at $s=0,1$ where the sound speed
\begin{equation}
C_S= \sqrt{K_M T_d}
\end{equation}
and $|M_s|$ is the Mach number, since $M_s$ will be allowed to take either sign. Normally $|M_s|=1$
so that $M_0=-1$ and $M_1$=1 where the subscript corresponds to value of~$s$.
The combined energy flux at each boundary has
\begin{equation}
|Q_{\|}|=m_i C_S N (\delta_e K_M T_e + \delta_i K_M T_i) \approx
m_i C_S N \delta K_M T_d
\end{equation}
if $\delta \approx \delta_e \approx \delta_i$. For definiteness, $\delta=\frac{1}{2}(\delta_e+\delta_i)$
will be assumed. Values of $\delta_\alpha$ from the literature imply $\delta=4.25$.
The energy flux factor~$g$ is chosen such that the easier model has the same energy flux.

For mathematical analysis, it is convenient to replace the source terms in \Eqr{sysn1}{syst1}
with equivalent fluxes, however this is unnecessary for computational purposes.
The forms the sources take are discussed below in \Sec{sources}. 

%\subsection{Fluid Equations with Sources}\label{sec:sys23fluid}
%\begin{eqnarray}\label{eq:sound}
%F^n(s)&=&\int_0^s \frac{S^n}{B} ds_{\|}\\
%F^u(s)&=&\int_0^s \frac{S^u}{m_i B} ds_{\|} \label{eq:souud}\\
%F^\mathcal{E}(s)&=&2\int_0^s\frac{S_i^\mathcal{E}+S_e^\mathcal{E}}{m_i B} ds_{\|} \label{eq:soutd}
%\end{eqnarray}
%and write
%
%The convention with respect to limits of integration is that they are specified in terms
%of parameterised length and to use the
%appropriate relation for $s_{\|}(s)$, thus in the case of \Eq{spar}, the
%lower limit of~$0$ corresponds to $s_{\|}(s)=-L$.
%%(Note that this approach is not normally pursued because most sources are proportional
%%to density~$N$ and the recombination sources to~$N^2$, ie. there is substantial feedback.
%Observing the identity
%\begin{equation}
%\frac{1}{B}\frac{\partial}{\partial s} Bn_B K_M T_d = \frac{\partial}{\partial s} (n_B K_M T_d)+ n_B K_M T_d
%\frac{\partial}{\partial s} (\ln B)
%\end{equation}
%and the frequent appearance of $n_B=N/B$, the governing equations in dimensional form become
%\begin{eqnarray}\label{eq:sysnd}
%\frac{\partial}{\partial t} n_B  + 
%\frac{\partial}{\partial s_{\|}} (n_B U )&=& \frac{\partial}{\partial s_{\|}} F^n\\
%\frac{\partial}{\partial t} (n_B U)+ 
%\frac{\partial}{\partial s_{\|}} (n_B U^2)&=&
%-\frac{\partial}{\partial s_{\|}} (n_B K_M T_d) +\frac{\partial}{\partial s_{\|}} F^u \label{eq:sysud}\\
%\frac{\partial}{\partial t}\left( \frac{3}{2}(n_B K_M T_d)+
%\frac{1}{2} (n_B U^2) \right) &+&\\
%\frac{\partial}{\partial s_{\|}} \left( \frac{5}{2}(n_B U K_M T_d) +
%\frac{1}{2} (n_B U^3) \right)  &=&
%-\frac{\partial}{\partial s_{\|}} F^Q +
%\frac{1}{2} \frac{\partial}{\partial s_{\|}} F^\mathcal{E} \label{eq:systd}
%\end{eqnarray}
%where the derivative of $\ln B$ has been neglected. The boundary conditions on $U$ are
%unchanged and
%\begin{equation}\label{eq:Qpd}
%|Q_{\|}|=
%m_i C_S n B \delta K_M T_d
%\end{equation}
%The \Eqr{sysnd}{systd} together with boundary condition \Eq{Qpd} are in units
%such that equivalence may easily be established with those of ref~\cite{Ha13Benc} (by 
%identifying $s$ with~$s_{\|}$).


\subsection{Explicit Sources}\label{sec:sources}
The above work considers the case where the source terms are regarded
as given, however it is worth describing the form of the additional sources that
may be at least locally important.
From ref~\cite{Ha13Benc}, the plasma sources are given by (with the convention that suffix~`n'
denotes neutral species)
\begin{eqnarray}
\label{eq:Sn} S^n&=&N_n N \langle\sigma v\rangle_{ION} - N^2 \langle\sigma v\rangle_{REC} +S^n_{\perp} \\
\label{eq:Su} S^u&=&N_n N \langle\sigma v\rangle_{ION} U_n - N^2 \langle\sigma v\rangle_{REC} U + N_n N (U_n-U) \langle\sigma v\rangle_{CX} \\
\label{eq:SE} S^\mathcal{E}&=&S^\mathcal{E}_i+S^\mathcal{E}_e \\
&=&N_n N \langle\sigma v\rangle_{ION} (\frac{3}{2} kT_n + \frac{1}{2} m_n U_n^2 -k I_H)\\
\nonumber &-& N^2 \langle\sigma v\rangle_{REC} (\frac{3}{2} kT_i + \frac{1}{2} m_i U^2 )\\
\nonumber &+&N_n N\langle\sigma v\rangle_{CX} \left(\frac{3}{2} k (T_n-T_i)  + \frac{1}{2} m_n (U_n^2-U^2)\right)\\
\nonumber &-&N_n N k Q_H +S^\mathcal{E}_{\perp i} +S^\mathcal{E}_{\perp e}
 \end{eqnarray}
Here suffix $\perp$ denotes the effectively given source terms arising from cross-field
contributions, suffices $ION$,
$REC$ and $CX$ denote respectively reaction rates~$\langle\sigma v\rangle$ for ionisation,
recombination and charge-exchange, $I_H$ is the Hydrogen reionisation potential,
and $Q_H$ is the cooling rate due to excitation. 

Since the sources appear in the analysis primarily as integrals starting at $s=0$,
study of \Eqr{Sn}{SE} concentrates on this region, where plasma velocity $U<0$ and neutral velocity $U_n>0$
with the two having approximately the same magnitude. There, \Eq{Sn} has only one negative
term, due to recombination, but from the cross-section data in ref~\cite{Ha13Benc}, this could
dominate only below~$2$\,eV. All terms in \Eq{Su} are positive near $s=0$ as the two velocities
reinforce. \Eq{SE} contains two terms which are always negative and an ionisation
term which is also negative below~$I_H/2\approx7$\,eV, thus for example, the cross-field source
terms~$S_{\perp i,e}$  must be positive for $S^\mathcal{E}>0$.

The sources of neutrals may be deduced from the ionisation and charge-exchange terms in \Eqr{Sn}{SE}, viz.
\begin{eqnarray}
\label{eq:Snn} S^n_n&=&-N_n N \langle\sigma v\rangle_{ION} +S^n_{\perp n} \\
\label{eq:Sun} S^u_n&=&-N_n N \langle\sigma v\rangle_{ION} U_n - N_n N (U_n-U) \langle\sigma v\rangle_{CX} + S^u_{\perp n} \\
\label{eq:SEn} S^\mathcal{E}_n&=&
-N_n N \langle\sigma v\rangle_{ION} (\frac{3}{2} kT_n + \frac{1}{2} m_n U_n^2 -k I_H)\\
&-&N_n \langle\sigma v\rangle_{CX} \left(\frac{3}{2} k (T_n-T_i)  + \frac{1}{2} m_n (U_n^2-U^2)\right)\\
&+&S^\mathcal{E}_{\perp n} 
\end{eqnarray}
The $S_{\perp n}$ terms are hard to quantify, but if these are neglected,
it is clear that $S^n_n<0$ and $S^u_n<0$
is the obverse of the positive plasma sources. Similarly it is likely that $S^\mathcal{E}_n<0$ if $S^\mathcal{E}>0$

\subsection{Dimensionless Coupled System}\label{sec:sys23coupled}
To proceed, it is helpful to make the preceding system of equations dimensionless,
by scaling $n$ with respect to~$N_0$, $T_d$ with respect to~$T_0$,
$U$ with respect to~$C_0$ and $B$ with respect to~$B_0$.
If the subscript~$0$ corresponds to the value of a variable at $s=0$, then
it is convenient to take $C_0=\sqrt{K_M T_0}$, implying $C_S=C_0$.

Suppose the sources are symmetrically disposed about $s_{\|}=0$ such
that $n(s)=n(1-s)$, $u(s)=-u(1-s)$ and~$T(s)=T(1-s)$.
Note that sheath boundaries will be plasma outflows implying $u(0)<0$.
The boundary conditions on $u$ are in general specified in terms of the (signed) Mach number
\begin{equation}
u(0)=u_0= M_0,\;\;\; u(1)=u_1 = M_1
\end{equation}
and the dimensionless outward energy flux (doubled) at the boundaries $s=0,1$ satisfies
\begin{equation}\label{eq:Qp}
|q_s|=2 \delta n_s T(s)
\end{equation}
Thus the boundary conditions on the plasma are completely specified by
setting field values at $s=0$ of
\begin{eqnarray}
\label{eq:bcpdn} n=1 \\
\label{eq:bcpdu} u=-1 \\
\label{eq:bcpdt} T=1
\end{eqnarray}
when a steady solution is possible in the absence of neutral source terms only if it is arranged so that
\begin{eqnarray}
\label{eq:intsn} \int s^n dx = 2 \\
\label{eq:intsu} \int s^u dx = 0 \\
\label{eq:intst} \int s^{\mathcal{E}} dx = 2(g+1) = 4 \delta
\end{eqnarray}

Working within the flux-tube geometry, the equations for neutral transport
take the same form as those used for plasma above,
however the boundary conditions are different.
They become at $s=0$, supposing that ${\sf T}(0)=\tau^2 T_0$ and masses~$m_n=m_i$, in dimensionless units,
\begin{eqnarray}
\label{eq:bcnn} {\sf n}=R_2=\frac{R_p}{\tau}|M_0| \\
\label{eq:bcnu} {\sf u}=-\tau \\
\label{eq:bcnt} {\sf T}=\tau^2 
\end{eqnarray}
where ${\sf n}$, ${\sf u}$ and ${\sf T}$ are neutral density, velocity and temperature made
dimensionless with respect to the same $N_0$ and $T_0$ as the corresponding plasma quantities,
and $R_p$ is the recycling coefficient. Note the usage of a sans-serif font to denote
dimensionless neutral species quantities, and that ${\sf n}$ does not include a factor of
$\tilde{b}=B/B_0$, as the neutral flux is not constrained by the flux tube.
%Moreover, if ${\sf T}=T_i$, then $T_0$ may be redefined so that $\tau^2=T_i/(T_i+T_e)$.


A system explicitly modelling the coupling between plasma and neutrals may be derived by making
non-dimensionless the sources set out in \Sec{sources}, and assuming $m_n=m_i$, ${\sf T}=T_i=\tau^2 T$,
giving $5$~equations for the evolution of plasma density, velocity, total temperature,
neutral density and neutral velocity:
\begin{eqnarray}\label{eq:syscn}
\epsilon_r \frac{\partial}{\partial t} n + 
\frac{\partial}{\partial s} (nu)&=& \sigma_I n {\sf n}+ s^n\\
\epsilon_r \frac{\partial}{\partial t} (nu)+ 
\frac{\partial}{\partial s} (nu^2+nT)&=&
\sigma_I n {\sf n} {\sf u} + \sigma_C n {\sf n} ({\sf u}-u)+ s^u \label{eq:syscu}\\
\epsilon_r \frac{\partial}{\partial t}\left( (g-2) nT+
 nu^2) \right) &+& \nonumber\\
\frac{\partial}{\partial s} \left( g nuT +
nu^3 \right) &=& \sigma_I n {\sf n} (3[\tau^2 T -T_H]+{\sf u}^2)
+\sigma_C n {\sf n} ({\sf u}^2-u^2) \nonumber\\
&-&\sigma_E n {\sf n}
+ s^\mathcal{E} \label{eq:sysct}\\
\epsilon_r \frac{\partial}{\partial t} {\sf n} + 
\frac{\partial}{\partial s} ({\sf n} {\sf u})&=& -\sigma_I n {\sf n}+s^n_n\label{eq:syscnn}\\
\epsilon_r \frac{\partial}{\partial t} ({\sf n} {\sf u})+ 
\frac{\partial}{\partial s} ({\sf n} {\sf u}^2+{\sf n} \tau^2 T)&=&
-\sigma_I n {\sf n} {\sf u} - \sigma_C n {\sf n} ({\sf u}-u) +s^u_n \label{eq:syscnu}
\end{eqnarray}
where $\epsilon_r=L_s/(t_0 C_0)$ with $t_0$ a characteristic timescale. The reaction rates
are made dimensionless by division by $C_0/(L_s N_0)$($\simeq 3.1 \times 10^{-16}\,m^3 s^{-1}$ for
representative parameter values), so that
\begin{eqnarray}\label{eq:csigma}
\sigma_I &=& \frac{\langle \sigma v \rangle_{ION}}{C_0/(L_s N_0)}\\
\sigma_C &=& \frac{\langle \sigma v \rangle_{CX}}{C_0/(L_s N_0)}\\
\sigma_E &=& \frac{2 Q_H}{C_0 T_0/(L_s N_0)}\\
T_H&=&\frac{2 I_H}{3 T_0} \simeq 9/T_0 \mbox{(in eV)}
\end{eqnarray}
where $I_H$ and $Q_H$ are as defined in ref~\cite{Ha13Benc}.
The source terms are made dimensionless by division by $N_0 C_0 \mathrm{Scale}/L_s$
where $\mathrm{Scale}$ is unity for \Eq{syscn} and  $m_i$ for \Eqs{syscu}{sysct}.
Spatially variable~$B$ is accounted for by identifying $n \rightarrow n/\tilde{b}$,
$s^{n,u,\mathcal{E}}\rightarrow s^{n,u,\mathcal{E}}/\tilde{b}$, and dividing
the term $\frac{\partial}{\partial s} (nT)$ in \Eq{syscu} by $\tilde{b}$
(or by taking $\frac{\partial}{\partial s} (\ln \tilde{b})$ to be negligible).

It is of interest to allow a stochastic (`turbulent') contribution to the terms~$s^{n,u,\mathcal{E}}$.

\subsection{Neutrals as 2-D or 3-D fluid}\label{sec:sys23neutrals}
The above \Eqs{syscnn}{syscnu} are a reduction to 1-D of the system
\begin{eqnarray}
\epsilon_r \frac{\partial}{\partial t} {\sf n} +
\nabla \cdot({\sf n} {\bf\sf u})&=& -\sigma_I n {\sf n} + s^n_n \label{eq:syvcnn}\\
\epsilon_r \frac{\partial}{\partial t} ({\sf n} {\bf\sf u})+
\nabla \cdot ({\sf n} {\bf\sf u} {\bf\sf u}) + \nabla {\sf n} {\sf T} &=&
-\sigma_I n {\sf n} {\bf\sf u} - \sigma_C n {\sf n} ({\bf\sf u}-\bf{u}) +s^u_n \label{eq:syvcnu}
\end{eqnarray}

\subsection{Uncertainty Quantification}\label{sec:sys23UQ}
%Stochastic sources like in Ben Sloman's report~\cite{Sl15MLMC}.
Polynomial chaos~(PC) refers to a situation whereby probability
functions are expanded as Hermite polynomials, and generalised
Polynomial chaos~(gPC) to expansions using Hermite and other
polynomial sets~$\{\Psi_j(\xi)\}$~\cite{Xi03Mode}.
Thus for example, suppose $\theta$ to denote a random event, and the number density field
to have the following representation in terms of a finite number~$P$ of such modes.
\begin{equation}\label{eq:TrepgPC}
n({\bf x},t, \theta) = \sum_{j=0}^{P_C} n_j({\bf x},t) \Psi_j \left(\xib(\theta)\right)
\end{equation}
where
$\{ n_j({\bf x},t)\}$ is the set of deterministic coefficients of the ``random trial basis",
ie.\ the set~$\{ \Psi_j \left(\xib(\theta)\right) \}$ where $\xib(\theta)$ 
is a multi-dimensional random variable with a specific 
probability distribution as a function of the random parameter~$0\leq\theta\leq 1$.
Note that the $\Psi_j$ are the set of multi-dimensional Hermite polynomials if $\xib$ is a vector.
Typically but not necessarily the $\xib(\theta)$ will be Gaussians.
%The Monte-Carlo algorithm can be thought of as a subcase of the above representation
%corresponding to the collocation procedure where the test basis
%is~$\Psi_j(\theta)=\delta(\theta-\theta_j)$
%where $\delta$ is the Kronecker delta function and $\theta_j$
%refers to an isolated random event~\cite{Xi03Mode}.
Expressions like~\Eq{TgPC} may be substituted into a governing
equation of say advection type for~$n$, and the result simplifies because spatial
operators do not interact with the random variables, then taking the inner product
with $\Psi_k \left(\xib(\theta)\right)$ yields
\begin{equation}\label{eq:TgPC}
\frac{\partial n_k}{\partial t}+  
\sum_{i=0}^{P_C} \sum_{j=0}^{P_C} e_{ijk} \frac{\partial u_i n_j}{\partial s}=0
\end{equation}
where $e_{ijk}$ is a weighted integral of triple products of $\Psi_i$.
Hence there are now $P_C$ equations instead of one for~$n$.

There is an alternative treatment by Polynomial Chaos Expansion (PCE) which is non-intrusive
and therefore preferred. It relies on projecting a set of randomly selected solutions onto
a Hermite or similar basis, see further description in ref~\cite[end \S\,2.1.1]{y2re313}.
