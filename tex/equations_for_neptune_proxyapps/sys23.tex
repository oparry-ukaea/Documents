\subsection{Plasma Equations}\label{sec:sys23plas}

Starting from the two-fluid model of Braginskii~~\cite{Br65Tranwarv},
a set of equations resembling those of classical (compressible) hydrodynamics
may be derived by summing Braginskii's equations for number density,
momentum and energy~\cite{Ha13Benc}.
Using the standard notation of \Sec{mathsymb}, introducing $T_d=T_i+T_e$,
neglecting the stress tensor terms (implicitly setting $\delta p_i=0$),
and assuming $B$ is independent of time, the resulting system is
\begin{eqnarray}\label{eq:sysn1}
\frac{\partial}{\partial t} (\frac{N}{B})+ 
\frac{\partial}{\partial s_{\|}} (\frac{N U}{B})&=&\frac{S^n}{B} \\
\frac{\partial}{\partial t} (\frac{m_i N U}{B})+ 
\frac{\partial}{\partial s_{\|}} (\frac{m_i N U^2}{B})&=&
-\frac{1}{B}\frac{\partial}{\partial s_{\|}} (p_i + p_e) +\frac{S^u}{B} \label{eq:sysu1}\\
\frac{\partial}{\partial t}\left( \frac{3}{2}(\frac{N k T_d}{B})+
\frac{1}{2} (\frac{m_i N U^2}{B}) \right) &+& \nonumber \\
\frac{\partial}{\partial s_{\|}} \left( \frac{5}{2}(\frac{N U k T_d}{B}) +
\frac{1}{2} (\frac{m_i N U^3}{B}) \right) &=&
-\frac{1}{B}\frac{\partial}{\partial s_{\|}} (q_{i\|} + q_{e\|}) +
\frac{(S_i^\mathcal{E}+S_e^\mathcal{E})}{m_i B} \label{eq:syst1}
\end{eqnarray}
%$U_d =L_s/t_0$ is a speed measuring the importance of the transient term,
where $s_{\|}$ is distance along the fieldline,
%temporarily the conduction term in $(q_{i\|} + q_{e\|})$ has been retained,
and all variables retain their physical dimension. (Some variables from~\cite{Ha13Benc}
have been promoted to capitals to 
indicate that they retain their physical dimensions.)
%For the case of a fieldline connecting two walls at~$s_{\|}=\pm L$,
%suppose dimensionless~$s$, $0\leq s\leq 1)$ parameterises distance
%\begin{equation}\label{eq:spar}
%s_{\|}=L(2s-1)
%\end{equation}
%It is therefore helpful to make fieldline length dimensionless in terms of the distance
%(assumed constant)
%\begin{equation}\label{eq:hs}
%L_s= \frac{d s_{\|}}{d  s}
%\end{equation}
%and so $L_s=2L$.
%The constant $K_{AM}$ is such that
%\begin{equation}
%K_{AM}= \frac{k_B}{m_i}\;\;\mbox{or}\;\;K_{AM}=\frac{|e|}{m_i}=\frac{|e|}{Am_u}
%\end{equation}
%where $k_B$ is Boltzmann's constant and $|e|$ is the unit of charge,
%depending whether T is measured in Kelvin or~$eV$,
Ion mass is defined as $m_i= Am_u$ where
$A$ is atomic mass of the ion and $m_u$ is the atomic mass unit. 

Note that in adding Eqs.(3) and~(4) of~\cite{Ha13Benc}, equipartition and collision terms cancel 
to give \Eq{syst1}. The perfect gas equation of state will be assumed, so that
\begin{equation}\label{eq:sysp}
p_i+p_e=N k T
\end{equation}

The thermal conduction fluxes are
\begin{equation}\label{eq:qei}
q_{\alpha\|}= -\kappa^\alpha_{\|} \frac{\partial k T_\alpha}{\partial s_{\|}},\;\;\;\alpha=e,\;i
\end{equation}
where the $\kappa^\alpha_{\|}$ take Braginskii values, see \Sec{bragin}, thus for usual situation
in which the electron conduction dominates
\begin{equation}\label{eq:qeisum}
q_{i\|} + q_{e\|}= \kappa^e_{\|} \frac{\partial k T_e}{\partial s_{\|}}
\end{equation}
and  $T_e$ has be expressed in terms of $T_d$, eg.\ $T_e=T_d/2$ or $T_e=(1-\tau^2)T_d$
in terms of an arbitrary $0<\tau<1$ that determines the ion temperature. To give an easier test
problem, the conduction term may be accounted for by augmenting the advective energy flux, $5/2\rightarrow g$.

The boundary conditions are that $|U|=|M_s| C_S$ at $s=0,1$ where the sound speed
\begin{equation}
C_S= \sqrt{k T_d/m_i}
\end{equation}
and $|M_s|$ is the Mach number, since $M_s$ will be allowed to take either sign. Normally $|M_s|=1$
so that $M_0=-1$ and $M_1$=1 where the subscript corresponds to value of~$s$.
The combined energy flux at each boundary has
\begin{equation}
|Q_{\|}|=\frac{1}{2}C_S N (\delta_e k T_e + \delta_i k T_i) \approx \frac{1}{2}
m_i C_S N \delta k T_d
\end{equation}
if $\delta \approx \delta_e \approx \delta_i$. For definiteness, $\delta=\frac{1}{2}(\delta_e+\delta_i)$
will be assumed. Values of $\delta_\alpha$ from the literature imply $\delta=4.25$.
The energy flux factor~$g$ is chosen such that the easier model has the same energy flux.

For mathematical analysis, it is convenient to replace the source terms in \Eqr{sysn1}{syst1}
with equivalent fluxes, however this is unnecessary for computational purposes.
The forms the sources take are discussed below in \Sec{sources}. 

%\subsection{Fluid Equations with Sources}\label{sec:sys23fluid}
%\begin{eqnarray}\label{eq:sound}
%F^n(s)&=&\int_0^s \frac{S^n}{B} ds_{\|}\\
%F^u(s)&=&\int_0^s \frac{S^u}{m_i B} ds_{\|} \label{eq:souud}\\
%F^\mathcal{E}(s)&=&2\int_0^s\frac{S_i^\mathcal{E}+S_e^\mathcal{E}}{m_i B} ds_{\|} \label{eq:soutd}
%\end{eqnarray}
%and write
%
%The convention with respect to limits of integration is that they are specified in terms
%of parameterised length and to use the
%appropriate relation for $s_{\|}(s)$, thus in the case of \Eq{spar}, the
%lower limit of~$0$ corresponds to $s_{\|}(s)=-L$.
%%(Note that this approach is not normally pursued because most sources are proportional
%%to density~$N$ and the recombination sources to~$N^2$, ie. there is substantial feedback.
%Observing the identity
%\begin{equation}
%\frac{1}{B}\frac{\partial}{\partial s} Bn_B K_M T_d = \frac{\partial}{\partial s} (n_B K_M T_d)+ n_B K_M T_d
%\frac{\partial}{\partial s} (\ln B)
%\end{equation}
%and the frequent appearance of $n_B=N/B$, the governing equations in dimensional form become
%\begin{eqnarray}\label{eq:sysnd}
%\frac{\partial}{\partial t} n_B  + 
%\frac{\partial}{\partial s_{\|}} (n_B U )&=& \frac{\partial}{\partial s_{\|}} F^n\\
%\frac{\partial}{\partial t} (n_B U)+ 
%\frac{\partial}{\partial s_{\|}} (n_B U^2)&=&
%-\frac{\partial}{\partial s_{\|}} (n_B K_M T_d) +\frac{\partial}{\partial s_{\|}} F^u \label{eq:sysud}\\
%\frac{\partial}{\partial t}\left( \frac{3}{2}(n_B K_M T_d)+
%\frac{1}{2} (n_B U^2) \right) &+&\\
%\frac{\partial}{\partial s_{\|}} \left( \frac{5}{2}(n_B U K_M T_d) +
%\frac{1}{2} (n_B U^3) \right)  &=&
%-\frac{\partial}{\partial s_{\|}} F^Q +
%\frac{1}{2} \frac{\partial}{\partial s_{\|}} F^\mathcal{E} \label{eq:systd}
%\end{eqnarray}
%where the derivative of $\ln B$ has been neglected. The boundary conditions on $U$ are
%unchanged and
%\begin{equation}\label{eq:Qpd}
%|Q_{\|}|=
%m_i C_S n B \delta K_M T_d
%\end{equation}
%The \Eqr{sysnd}{systd} together with boundary condition \Eq{Qpd} are in units
%such that equivalence may easily be established with those of ref~\cite{Ha13Benc} (by 
%identifying $s$ with~$s_{\|}$).


\subsection{Explicit Sources}\label{sec:sources}
The above work considers the case where the source terms are regarded
as given, however it is worth describing the form of the additional sources that
may be at least locally important.
From ref~\cite{Ha13Benc}, the plasma sources are given by (with the convention that suffix~`n'
denotes neutral species)
\begin{eqnarray}
\label{eq:Sn} S^n&=&N_n N \langle\sigma v\rangle_{ION} - N^2 \langle\sigma v\rangle_{REC} +S^n_{\perp} \\
\label{eq:Su} \frac{S^u}{m_i} &=&N_n N \langle\sigma v\rangle_{ION} U_n - N^2 \langle\sigma v\rangle_{REC} U + N_n N (U_n-U) \langle\sigma v\rangle_{CX} \\
\label{eq:SE} S^\mathcal{E}&=&S^\mathcal{E}_i+S^\mathcal{E}_e \\
&=&N_n N \langle\sigma v\rangle_{ION} (\frac{3}{2} kT_n + \frac{1}{2} m_n U_n^2 -k I_H)\\
\nonumber &-& N^2 \langle\sigma v\rangle_{REC} (\frac{3}{2} kT_i + \frac{1}{2} m_i U^2 )\\
\nonumber &+&N_n N\langle\sigma v\rangle_{CX} \left(\frac{3}{2} k (T_n-T_i)  + \frac{1}{2} m_n (U_n^2-U^2)\right)\\
\nonumber &-&N_n N k Q_H +S^\mathcal{E}_{\perp i} +S^\mathcal{E}_{\perp e}
 \end{eqnarray}
Here suffix $\perp$ denotes the effectively given source terms arising from cross-field
contributions, suffices $ION$,
$REC$ and $CX$ denote respectively reaction rates~$\langle\sigma v\rangle$ for ionisation,
recombination and charge-exchange, $I_H$ is the Hydrogen reionisation potential,
and $Q_H$ is the cooling rate due to excitation. 

Since the sources appear in the analysis primarily as integrals starting at $s=0$,
study of \Eqr{Sn}{SE} concentrates on this region, where plasma velocity $U<0$ and neutral velocity $U_n>0$
with the two having approximately the same magnitude. There, \Eq{Sn} has only one negative
term, due to recombination, but from the cross-section data in ref~\cite{Ha13Benc}, this could
dominate only below~$2$\,eV. All terms in \Eq{Su} are positive near $s=0$ as the two velocities
reinforce. \Eq{SE} contains two terms which are always negative and an ionisation
term which is also negative below~$I_H/2\approx7$\,eV, thus for example, the cross-field source
terms~$S_{\perp i,e}$  must be positive for $S^\mathcal{E}>0$ in steady state.

The sources of neutrals may be deduced from the ionisation and charge-exchange terms in \Eqr{Sn}{SE}, viz.
\begin{eqnarray}
\label{eq:Snn} S^n_n&=&-N_n N \langle\sigma v\rangle_{ION} +S^n_{\perp n} \\
\label{eq:Sun} \frac{S^u_n}{m_n} &=&-N_n N \langle\sigma v\rangle_{ION} U_n - N_n N (U_n-U) \langle\sigma v\rangle_{CX} + S^u_{\perp n} \\
\label{eq:SEn} S^\mathcal{E}_n&=&
-N_n N \langle\sigma v\rangle_{ION} (\frac{3}{2} kT_n + \frac{1}{2} m_n U_n^2 -k I_H)\\
&-&N_n \langle\sigma v\rangle_{CX} \left(\frac{3}{2} k (T_n-T_i)  + \frac{1}{2} m_n (U_n^2-U^2)\right)\\
&+&S^\mathcal{E}_{\perp n} 
\end{eqnarray}
The $S_{\perp n}$ terms are hard to quantify, but if these are neglected,
it is clear that $S^n_n<0$ and $S^u_n<0$
is the obverse of the positive plasma sources. Similarly it is likely that $S^\mathcal{E}_n<0$ if $S^\mathcal{E}>0$

The boundary conditions on the neutrals~\cite[Table\,4]{Ha13Benc} are
(1) that the flux of neutrals is set by recycling of the plasma, so
\begin{equation}\label{eq:bcneutflux}
N_n U_n = - R_p N U
\end{equation}
where $R_p$ is the recycling coefficient.
and (2) of close to sonic outflow
\begin{equation}\label{eq:bcneutspeed}
U_n ={\sf M}_0\sqrt\frac{k T_n}{m_n}
\end{equation}
where ${\sf M}_0$ is the signed Mach number for the neutrals, and $T_n$ is the neutral temperature in the
volume, which assumed constant throughout a calculation.

\subsubsection{Symmetries and Constraints}\label{sec:constraint}
Solutions that could be used for testing purposes are described in the separate dedicated document.
Here it is briefly noted that symmetry could be used to test code validity. The model
supports a solution symmetric about the domain mid-point in density and temperature
(antisymmetric in flow velocity), provided any applied sources have the corresponding symmetries.
There are also point relations which must be satisfied at the midpoint.

In steady state, there is the physical, integrated constraint that sources must balance total
fluxes of plasma and neutral mass across the boundaries.
Further, in absence of diffusive terms, if the sources vanish at the boundaries
then steady-state has the additional constraint of boundary conditions
of zero gradient (Neumann conditions). Care is required though, as this will not generally be
true since the more realistic representation of the sources in the coupled system
typically leads to non-zero values at the boundaries.
