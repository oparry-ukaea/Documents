\subsection{System 2-6n: Two-fluid Plasma Model}\label{sec:2fluid}
The model system must be solved coupled to the Maxwell equations
for the elecromagnetic fields {\bf E} and {\bf B}
%%%%%%%%%%%%%%%%%%%%%%%%%%%%%%%%%%%%%%%%%%%%%%%%%%%%%%%%%%%%%%%%%%%%%%%%%%%%%%%%%%%%%%
\begin{equation}
\label{eq:bdot} 
\nabla \times {\bf E} = - \frac{\partial {\bf B}} {\partial t}
\end{equation}
\begin{equation}
\label{eq:dive} 
\epsilon_0 \nabla \cdot {\bf E} = \rho_c
\end{equation}
%%%%%%%%%%%%%%%%%%%%%%%%%%%%%%%%%%%%%%%%%%%%%%%%%%%%%%%%%%%%%%%%%%%%%%%%%%%%%%%%%%%%%%
%%%%%%%%%%%%%%%%%%%%%%%%%%%%%%%%%%%%%%%%%%%%%%%%%%%%%%%%%%%%%%%%%%%%%%%%%%%%%%%%%%%%%%
\begin{equation}
\label{eq:edot}
\nabla \times {\bf B} = \mu_0 {\bf j} +\frac{1}{c^2} \frac{\partial {\bf E}} {\partial t}
\end{equation}
\begin{equation}
\label{eq:divb}
\nabla\cdot {\bf B} = 0   
\end{equation}
%%%%%%%%%%%%%%%%%%%%%%%%%%%%%%%%%%%%%%%%%%%%%%%%%%%%%%%%%%%%%%%%%%%%%%%%%%%%%%%%%%%%%%
where $\epsilon_0$ and $\mu_0$ are the permittivity and permeability 
respectively of free space. For a two species plasma consisting of ions
for which $\alpha=i$ and electrons for which $\alpha=e$, the charge density $\rho_c$ and 
current density {\bf j} are defined as follows
%%%%%%%%%%%%%%%%%%%%%%%%%%%%%%%%%%%%%%%%%%%%%%%%%%%%%%%%%%%%%%%%%%%%%%%%%%%%%%%%%%%%%%
\begin{equation}
\label{eq:j}
{\bf j} =  n_i q_i {\bf v}_i + n_e q_e {\bf v}_e
\end{equation}
\begin{equation}
\label{eq:rhoc}
\rho_c =  n_i q_i  + n_e q_e 
\end{equation}
%%%%%%%%%%%%%%%%%%%%%%%%%%%%%%%%%%%%%%%%%%%%%%%%%%%%%%%%%%%%%%%%%%%%%%%%%%%%%%%%%%%%%%
Integrating the Boltzmann equation over velocity space gives
an equation for the  zeroth moment of $f = f_\alpha $ or number density $n_\alpha  = \int f d{\bf 
v}$, in terms of the first moment of $f$, fluid velocity ${\bf v}_\alpha  = \int f {\bf v}
d{\bf v}$, viz.
%%%%%%%%%%%%%%%%%%%%%%%%%%%%%%%%%%%%%%%%%%%%%%%%%%%%%%%%%%%%%%%%%%%%%%%%%%%%%%%%%%%%%%
\label{eq:e2.6}
\begin{equation}
\frac{\partial n_\alpha } {\partial t} + \nabla \cdot (n_\alpha {\bf v}_\alpha ) = S^n_\alpha .
\end{equation}
%%%%%%%%%%%%%%%%%%%%%%%%%%%%%%%%%%%%%%%%%%%%%%%%%%%%%%%%%%%%%%%%%%%%%%%%%%%%%%%%%%%%%%
where $S_\alpha $ is a source term.

A velocity weighted integral of the Boltzmann equation 
gives an equation for ${\bf v}_\alpha $, viz.
%%%%%%%%%%%%%%%%%%%%%%%%%%%%%%%%%%%%%%%%%%%%%%%%%%%%%%%%%%%%%%%%%%%%%%%%%%%%%%%%%%%%%%
\label{eq:e2.7}
\begin{equation}
\frac{\partial} {\partial t} (m_\alpha n_\alpha {\bf v}_\alpha ) +
\nabla  \cdot ( {\bf P}_\alpha  +m_\alpha  n_\alpha  {\bf v}_\alpha {\bf v}_\alpha  )
- q_\alpha  n_\alpha  ({\bf E} + {\bf v} \times {\bf B}) = 
{\bf F}_\alpha ^{coll},
\end{equation}
%%%%%%%%%%%%%%%%%%%%%%%%%%%%%%%%%%%%%%%%%%%%%%%%%%%%%%%%%%%%%%%%%%%%%%%%%%%%%%%%%%%%%%
that involves second moments of $f$, namely the pressure tensor
%%%%%%%%%%%%%%%%%%%%%%%%%%%%%%%%%%%%%%%%%%%%%%%%%%%%%%%%%%%%%%%%%%%%%%%%%%%%%%%%%%%%%%
\label{eq:e2.8}
\begin{equation}
{\bf P}_\alpha  =
\int m_\alpha  f
({\bf v} - {\bf u_\alpha })
({\bf v} - {\bf u_\alpha })
d{\bf v}
= p_\alpha  {\bf I} + {\bf \pi}_\alpha ,
\end{equation}
%%%%%%%%%%%%%%%%%%%%%%%%%%%%%%%%%%%%%%%%%%%%%%%%%%%%%%%%%%%%%%%%%%%%%%%%%%%%%%%%%%%%%%
where $p_\alpha $ is the scalar pressure (${\bf I}$ is the unit tensor) and 
${\bf \pi}_\alpha $ is the stress tensor for species $\alpha$.  Weighting the 
integral of the Boltzmann equation by the kinetic energy gives the energy equation.
%%%%%%%%%%%%%%%%%%%%%%%%%%%%%%%%%%%%%%%%%%%%%%%%%%%%%%%%%%%%%%%%%%%%%%%%%%%%%%%%%%%%%%
\label{eq:e2.9}
\begin{equation}
\frac{\partial}{\partial t} (\frac{3}{2}p_\alpha +\frac{1}{2} m_\alpha n_\alpha {\bf v}_\alpha ^2) + \nabla \cdot
\left(\left[
\frac {1}{2} m_\alpha n_\alpha {\bf v}_\alpha ^2 + \frac{5}{2} p_\alpha \right]
{\bf v}_\alpha  + {\bf \pi}_\alpha  \cdot {\bf v}_\alpha  + {\bf q}_\alpha \right)
- q_\alpha  n_\alpha  {\bf E} \cdot {\bf v}_\alpha  = \epsilon_\alpha ^{coll},
\end{equation}
%%%%%%%%%%%%%%%%%%%%%%%%%%%%%%%%%%%%%%%%%%%%%%%%%%%%%%%%%%%%%%%%%%%%%%%%%%%%%%%%%%%%%%
where the heat flux ${\bf q}_\alpha $ is a third order moment of $f$.  Evidently, 
equations involving higher order moments may be constructed, but it is 
customary to seek a closure at this level.  The usual assumption is 
that $f$ is close to a Maxwellian distribution at each position, viz.
%%%%%%%%%%%%%%%%%%%%%%%%%%%%%%%%%%%%%%%%%%%%%%%%%%%%%%%%%%%%%%%%%%%%%%%%%%%%%%%%%%%%%%
\label{eq:e2.10}
\begin{equation}
f({\bf v}) \simeq \exp (-
\frac{1}{2} m{\bf v}^2/[k_BT_\alpha]),
\end{equation}
%%%%%%%%%%%%%%%%%%%%%%%%%%%%%%%%%%%%%%%%%%%%%%%%%%%%%%%%%%%%%%%%%%%%%%%%%%%%%%%%%%%%%%
where $T_\alpha $ is species temperature and $k_B$ is Boltzmann's constant.  It 
follows that
%%%%%%%%%%%%%%%%%%%%%%%%%%%%%%%%%%%%%%%%%%%%%%%%%%%%%%%%%%%%%%%%%%%%%%%%%%%%%%%%%%%%%%
\label{eq:e2.11}
\begin{equation}
p_\alpha = n_\alpha k_BT_\alpha
\end{equation}
%%%%%%%%%%%%%%%%%%%%%%%%%%%%%%%%%%%%%%%%%%%%%%%%%%%%%%%%%%%%%%%%%%%%%%%%%%%%%%%%%%%%%%
and for the collision operator appropriate to interactions between two 
species $\alpha=i$ and $e$ corresponding to ions and electrons 
respectively, expressions for ${\bf F}_\alpha^{coll}, \epsilon_\alpha^{coll}, 
{\bf \pi}_\alpha$ and ${\bf q}_\alpha$ are available from Braginskii (1965), see also Epperlein
and Haines (1985).  Crudely speaking, the effects of ${\bf \pi}_\alpha$ and ${\bf q}_\alpha$ are
to introduce diffusion into \Eq{e2.7}  and \Eq{e2.9}, e.g. the term $\nabla \cdot (\kappa_\alpha \nabla
T_\alpha)$ appears in the  latter.  For all their complexity the Braginskii or classical 
transport equations for  given ${\bf E}$ and ${\bf B}$, are basically of 
advection-diffusion type.  It also worth remarking that although  ions and electrons
are separately close to Maxwellian in a tokamak,  they generally have different
temperatures.  The equipartition term  (part of $\epsilon_\alpha$) representing
interspecies collisions that drive $T_i$ towards $T_e$, is often significant  but not
dominant.

Further simplification is required because the two-fluid equations are 
stiff in several senses.  Firstly, the momentum equations for ${\bf 
v}_e$ and ${\bf v}_i$ have timescales in the ratio $m_i/m_e \approx 1836$ even for
Hydrogen ions, and secondly the electrostatic forces  are relatively enormous.  The
latter may be seen from Gauss' Law  if the  electrostatic potential is introduced so that
${\bf E} = - \nabla  \phi$, then the ratio
%%%%%%%%%%%%%%%%%%%%%%%%%%%%%%%%%%%%%%%%%%%%%%%%%%%%%%%%%%%%%%%%%%%%%%%%%%%%%%%%%%%%%%
\label{eq:e2.12}
\begin{equation}
\frac{\rho_c}{\epsilon_o \nabla \cdot {\bf E}}
=
\frac{L^2 \sum_\alpha q_\alpha n_\alpha}{\epsilon_o \Delta \phi_o}
\end{equation}
%%%%%%%%%%%%%%%%%%%%%%%%%%%%%%%%%%%%%%%%%%%%%%%%%%%%%%%%%%%%%%%%%%%%%%%%%%%%%%%%%%%%%%
where $\Delta \phi_o$ is a typical fluctuation level in the potential, 
assumed to have a length-scale L comparable to device dimensions.  Writing $\sum_\alpha  q_\alpha 
n_\alpha  = en_o \Delta  n$, \Eq{e2.12} can be reexpressed as
%%%%%%%%%%%%%%%%%%%%%%%%%%%%%%%%%%%%%%%%%%%%%%%%%%%%%%%%%%%%%%%%%%%%%%%%%%%%%%%%%%%%%%
\label{eq:e2.13}
\begin{equation}
\frac{\rho_c}{\epsilon_o \nabla \cdot {\bf E}} =
\frac{n_o e^2}{\epsilon_o m_e}
\frac{L^2 \Delta n}{(e \Delta \phi_o/m_e)} = \omega^2_{pe}
\frac{L^2 \Delta n}
{(\frac{1}{2} v_o^2)}.
\end{equation}
%%%%%%%%%%%%%%%%%%%%%%%%%%%%%%%%%%%%%%%%%%%%%%%%%%%%%%%%%%%%%%%%%%%%%%%%%%%%%%%%%%%%%%
The quantity $\omega_{pe}$ is the plasma frequency, which for a reference density $n_o
=  10^{20} m^{-3}$ is $5.6 \times 10^{11}s^{-1}$.  If the speed $v_o$ is 
identified with the electron sound speed $v_{the}$ then $\omega_{pe}/v_{the} =
1/\lambda_D$, where $\lambda_D$ is known as the Debye length, and since clearly
$\lambda_D << 1$, it follows since all ratios in \Eq{e2.13} must be unity, that tokamak
plasmas are forced by the electric field to be  quasi-neutral, meaning $\Delta n <<
1$.  Scaling $\Delta n$  to be of order unity results in a stiffness
of $L^2/\lambda^2_D$.

