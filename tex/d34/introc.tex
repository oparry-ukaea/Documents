Drawing on the lessons learnt from earlier \nep\ work, especially ref~\cite{y2re312}, not only will the
software will be opensource, but in the interests of encouraging as wide a possible range of developers,
\nep\ will be treated as a community project with maximal transparency regarded decision-making, resource allocation etc.
Successful examples of such projects set out principles that community members are expected to adhere to,
and guidelines for them to follow. This amounts to significantly more detail both concerning the
management of the development and the production of the software, than appears in the \nep\ Science Plan
and Charter~\cite{sciplan,charter} and the subsequent initial reports~\cite{y1re311,y1re331}. 

The present report looks in detail at a number of sources recommending what documents are required
for the successful conduct of a software development project. Following from the report~\cite{y1re311},
the first of these is the book chapter by Spencer Smith~\cite{Sm17Rati},
as specifically directed to \emph{scientific} software.
The conclusion of the report~\cite[\S\,3]{y2re312} draws attention to Sommerville~\cite{sommerville10}
who has relevant material dispersed throughout the book and Hewitt~\cite{hewittexc}, although these
are aimed at a wider, more commercially-oriented range of developments. Focussing on science,
specifically space applications
there is the European Space Agency~(ESA) standards document~\cite{ecss40exc} which in
a ``lite" form has proved successful in two 
UKAEA developments this century, viz.\ \F{FISPACT}~\cite{Su17FISP} and \F{SMARDDA}~\cite{Wa15}.

None of these sources is entirely suitable for \nep. 
The proposed ``faking" of documents by Smith seems unsatisfactory. Hurried writing  is suggested
by the inconsistency
between his statements that no-one likes documenting software and his recommendations that more
documents be produced, a suggestion supported by the fact that the proposed division
of material between various documents seems unsatisfactory in some places, eg.\ where the
functional and nonfunctional requirements are combined together in one section of a document
entitled the ``Software Requirements Specification". As a text-book, Sommerville offers a multitude
of possibilities, but few are relevant to open-source scientific software; therefore Hewitt scores
by being more focussed and by an emphasis on awareness of the surrounding business environment.
The ESA approach does not meet \nep\ needs in that it assumes a single expert customer
for the project, and anyway
the full standard, intended for flight-critical software, is arguably too onerous, where the
``lite" version is more appropriate for a contractor with a single worker.

In addition to the points in the preceding paragraph, none of the sources fully addresses the issue
that scientist developers do not like producing documentation. Ideally, material should be written up
only once to be presented to developers and users of whatever level of competence as appropriate. This
is probably best addressed by producing documentation as web pages, as indeed indicated by both Spencer Smith
and Hewitt.
%Moreover to minimise documentation writing for \nep, the arrangement of these web pages
%into a public website for the code  ought to be considered at the outset. To this end, the structure
%of the plasmapy website~\cite{plasmapywebsite} has been studied, particularly the documentation.

The main \Sec{taskwork} provides a concordance of the documents from the
texts~\cite{hewitt, Sm17Rati, sommerville10} in \Sec{concord}, then gives more details
as to what information should be provided in each in \Sec{detailedinfo}. It is followed by
discussion of how best to format the documentation as a website in \Sec{toweb}.



